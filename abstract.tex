\chapter*{Abstract}

\par Low Mass X-Ray Binaries (LMXBs) are systems in which a black hole or neutron star accretes matter from a stellar binary companion.  The accreted matter forms a disk of material around the compact object, known as an accretion disk.  The X-ray properties of LMXBs show strong variability over timescales ranging from milliseconds to decades.  Many of these types of variability are tied to the extreme environment of the inner accretion disk, and hence an understanding of this behaviour is key to understanding how matter behaves in such an environment.  GRS 1915+105 and MXB 1730-335 (also known as the Rapid Burster) are two LMXBs which show particularly unusual variability.  GRS 1915+105 shows a large number of distinct `classes' of second-to-minute scale variability, consisting of repeated patterns of dips and flares.  The Rapid Burster on the other hand shows `Type II X-ray Bursts'; second-to-minute scale increases in X-ray intensity with a sudden onset and a slower decay.  For many years both of these objects were thought to be unique amongst all known LMXBs.  More recently, two new objects, IGR J17091-3624 and GRO J1744-28 (also known as the Bursting Pulsar) have been shown to display similar behaviour to those seen in GRS 1915+105 and the Rapid Burster respectively.

\par In this thesis, I first present a new framework with which to classify variability seen in IGR J17091-3624.  Using my set of independent variability classes constructed for IGR J17091-3624, I perform a study of the similarities and differences between this source and GRS 1915+105 to better constrain their underlying physics.  In GRS 1915, hard X-ray emission lags soft X-ray emission in all variability classes; in IGR J17091, I find that the sign of this lag is different in variability classes.  Additionally, while GRS 1915+105 accretes at close to its Eddington Limit, I find that IGR J17091-3624 accretes at only $\sim5$--33\% of its Eddington Limit.  With these results I rule out any models which require near-Eddington accretion or hard corona reacting to the disk.  I also perform a study of the variability seen in the Bursting Pulsar.  I find that the flaring behaviour in the Bursting Pulsar is significantly more complex than in the Rapid Burster, consisting of at least 4 separate phenomena which may have separate physical origins.  One of these phenomena, `Structured Bursting', consists of patterns of flares and dips which are similar to those seen in GRS 1915+105 and IGR J17091-3624.  I compare these two types of variability and discuss the possibility that they are caused by the same physical instability.  I also present the alternative hypothesis that Structured Bursting is a manifestation of `hiccup' variability; a bimodal flickering of the accretion rate seen in systems approaching the `propeller' regime.