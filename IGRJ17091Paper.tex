\chapter{Variability in IGR J17091-3624: Classification}

\label{ch:IGR}

\epigraph{\textit{Song and call are useful aids to identification, and reference is made to vocalisation for each species.}}{Paul Sterry -- \textit{Collins Guide to British Birds}}

\vspace{1cm}

\par\noindent Accounting for the unusual X-ray variability\index{Variability} observed in LMXBs is required for a complete understanding of the physics of matter in their accretion disks\index{Accretion disk}.  The first step is to describe and categorise the types of variability in these objects, and to look for similarities and differences which may shed light on their physical origins.
\par In 2000, \citeauthor{Belloni_GRS_MI} performed a complete model-independent analysis of variability classes\index{Variability class} in GRS 1915\index{GRS 1915+105}.  This work highlighted the breadth and diversity of variability\index{Variability} in GRS 1915, and allowed these authors to search for features common to all variability classes.  For example, \citet{Belloni_GRS_MI} found that every variability class can be expressed as a pattern of transitions between three quasi-stable phenomenological states.
\par Previous works have noted that some of the variability classes\index{Variability} seen in IGR J17091\index{IGR J17091-3624} appear very similar to those seen in GRS 1915 (e.g. \citealp{Altamirano_IGR_FH, Zhang_IGR}).  However, although $\rho$-like\indexrho\ classes in the two objects both show lags between hard and soft X-rays photons\index{Hard lag}, these lags appear to possess different signs \citep{Altamirano_IGR_FH}.  Additionally, at least two variability classes have been reported in IGR J17091 which have not yet been reported in GRS 1915 \citep{Pahari_IGRClasses}.  Previous works have described some of the behaviour seen in IGR J17091 in the context of the variability classes described by \citealt{Belloni_GRS_MI} for GRS 1915 (e.g. \citealp{Altamirano_IGR_FH,Pahari_RhoDiff}).  To further explore the comparison between GRS 1915 and IGR J17091, here I perform the first comprehensive model-independent analysis of variability classes in IGR J17091 using the complete set of \indexrxte\rxte\ data taken of the 2011-2013 outburst\index{Outburst} of the object.  I also use data from all other X-ray missions that observed the source during this time to analyse the long-term evolution of the outburst.
\par \textbf{The results I present in this chapter have been published as \citet{IGR}.}

\section{Data and Data Analysis}

\label{sec:dex}

\par In this chapter, I report data from \indexrxte\rxte, \indexintegral\textit{INTEGRAL}, \indexswift\textit{Swift}, \indexchandra\textit{Chandra}, \indexxmm\textit{XMM-Newton} and \indexsuzaku\textit{Suzaku} covering the 2011-2013 outburst of IGR J17091.  Unless stated otherwise, all errors are quoted at the 1$\sigma$ level.
\par In Figure \ref{fig:allmissions} I present long-term lightcurves from \rxte ,  \textit{INTEGRAL} and \textit{Swift} to show the behaviour of the source during this outburst.  I indicate when during the outburst \textit{Chandra}, \textit{XMM-Newton} and \textit{Suzaku} observations were made.

\begin{figure}
    \includegraphics[width=\columnwidth, trim = {0.75cm 1.0cm 1.0cm 0.8cm},clip]{images/allmis.eps}
    \captionsetup{singlelinecheck=off}
    \caption[Lightcurves of IGR J17091-3624, from a number of instruments, during its 2011-2013 outburst.]{\indexrxte\indexpca\rxte\ /PCA (Panel a), \indexswift\indexxrt\textit{Swift/XRT} (Panel b), \indexbat\textit{Swift/BAT} (Panel b) and \indexintegral\indexibis\textit{INTEGRAL}/IBIS (Panel d) lightcurves\index{Lightcurve} of IGR J17091-3624\index{IGR J17091-3624} during its 2011-2013 outburst.  Arrows mark times at which \indexxmm\textit{XMM-Newton} (blue), \indexchandra\textit{Chandra} (red) or \indexsuzaku\textit{Suzaku} (magenta) observed IGR J17091-3624.  The cyan line represents MJD 55963, the approximate time IGR J17091-3624 transitions from the soft\index{High/Soft state} to the hard\index{Low/Hard state} state \citep{Drave_Return}.  \rxte /PCA \citep{Jahoda_PCA} data are for the 2--16\,keV energy band and taken from \citep{Altamirano_IGR_FH}, \textit{Swift/BAT} \citep{Barthelmy_BAT} data are for 15--50\,keV, \textit{Swift/XRT} \citep{Burrows_XRT}  data are for 0.3--10\,keV and \textit{INTEGRAL/ISGRI} \citep{Ubertini_IBIS} data are for 20-40\,keV.  Note that the data from \textit{Swift/XRT} (Panel B) are shown with a logarithmic $y$-axis to better show the late time progression of the outburst.  Data points are coloured according to the observing mode used.  The \textit{Swift/XRT} data from times later than MJD 56422 are shown to a different scale to better represent the post-outburst evolution of the source.  All data are presented in 1 day bins, except for data from \textit{Swift/BAT} which is presented in 4 day bins.  See also Figure \ref{fig:WhereCls}, in which data from \rxte \textit{/PCA} is presented on a smaller scale.  The Crab\index{Crab nebula} count rates used to normalise these data were 2300 cts s$^{-1}$ PCU$^{-1}$, 747.5 cts s$^{-1}$, 0.214 cts s$^{-1}$ and 183.5 cts s$^{-1}$ for \rxte , \textit{Swift/XRT}, \textit{Swift/BAT} and \textit{INTEGRAL/ISGRI} respectively.  \rxte\ data have not been corrected for the 25' offset to avoid contamination from GX 349+2\index{GX 349+2}, and for all instruments \textsf{D.A.} and I implicitly assume that IGR J17091 presents a Crab-like spectrum.}
   \label{fig:allmissions}
\end{figure}

\subsection{\rxte}

\label{sec:XTEDA}

\par For this variability study, I focus on the data from \indexrxte\indexpca\textit{RXTE}/PCA.  I analysed all PCA observations of IGR J17091\index{IGR J17091-3624} during 2011, corresponding to ObsIDs\footnote{Observations IDs.} 96065-03, 96103-01 and 96420-01.  The observations taken for proposals 96065-03 and 96103-01 were contaminated by the nearby X-ray source GX 349+2\index{GX 349+2} \citep{Altamirano_IGR_FH,Rodriguez_Contamination}.  As such I only use observations performed for proposal 96420-01, corresponding to a total of 243 orbits from 215 separate observations.  This in turn corresponds to 470\,ks of data, which is $\sim2$\% of \rxte 's operational time over the duration of the observation period.  These were offset by 25' such that GX 349+2 was not in the $1^\circ$ \textit{PCA} field of view.  \rxte\ was decommissioned during a period of Sun constraint centred on MJD\footnote{Modified Julian Date: the number of days since 0h00, November 17, 1858.} 55907, and hence the last observation of IGR J17091 was taken on MJD 55879.
\par I extracted data from the native \texttt{FITS}\index{FITS@\texttt{FITS}} format using my own \texttt{PANTHEON}\index{PANTHEON@\texttt{PANTHEON}} software (presented in Appendix \ref{app:PAN}).  To perform medium- to high-frequency ($\gtrsim1$\,Hz) timing analysis, I merged files formatted in PCA's `Good Xenon'\indexgx\ data mode and extracted their data at the maximum time resolution ($\sim9.5\times10^{-7}$ s) without accounting for the background.  I divided these data into 128\,s segments as this allowed us to reach frequencies below $\sim0.015$\,Hz, partly sampling the high amplitude quasi-periodic flaring behaviour seen in many classes.  Using the Fast Fourier Transform (FFT)\index{Fast Fourier transform}, I produced the power spectrum\index{Fourier analysis} of each segment separately.  I then averaged these spectra to create a one co-added Power Density Spectrum (PDS) for each observation.
\par For low-frequency ($\leq1$\,Hz) timing and correlated spectral/timing analysis, I rebinned the data to 0.5\,s and normalised count rates by the number of proportional counters (PCUs) active in each observation.  My choice of 1\,Hz allows us to analyse high amplitude flaring\index{Flare} behaviour (seen at frequencies $\lesssim0.5$\,Hz) separately from the lower-amplitude behaviour seen at $\gtrsim5$\,Hz.
\par I split the data into three energy bands: A (\textit{PCA} channels 0--14, $\sim2$--$6$\,keV), B (\textit{PCA} channels 15--35, $\sim6$--$16$\,keV) and C (\textit{PCA} channels 36--255, $\sim16$--$60$\,keV).  I chose these energy bands to be consistent with the energy bands used by the model-independent classification of variability classes of GRS 1915\index{GRS 1915+105} in \citet{Belloni_GRS_MI}.  For each of the energy-filtered lightcurves produced I estimated background using \texttt{pcabackest} from the \texttt{FTOOLS} package \citep{Blackburn_FTools} with the \textit{PCA} faint source background model\footnote{\url{http://heasarc.gsfc.nasa.gov/FTP/xte/calib\_data/pca\_bkgd/Faint/pca\_bkgd\_cmfaintl7\_eMv20051128.mdl}}\index{Background subtraction}. In all observations, I found that counts in the C band were consistent with background.  I then created Lightcurves $L_A$ and $L_B$ from background-subtracted photons counted in the A and B bands respectively.  I used these lightcurves to define the full-band lightcurve ($L_T=L_A+L_B$) and the soft colour\index{Colour} ($C_1=L_B/L_A$) of each observation.  To complement the Fourier spectra, I also constructed Generalised Lomb-Scargle Periodograms\index{Lomb-Scargle periodogram} of $L_T$ from each dataset, a modified version of the standard Lomb-Scargle periodogram \citep{Lomb_LombScargle, Scargle_LombScargle} that takes into account errors in the dataset \citep{Irwin_LombScargle}.  Using the Lomb-Scargle periodogram instead of the Fourier periodogram here allows us to sample the low-frequency behaviour of lightcurves with data gaps.  This is important, for example, in lightcurves which show two populations of flares, as it allows each population to be studied independently by cropping the other from the lightcurve.
\par I also used data from \citealt{Altamirano_IGR_FH} to sample the long-term colour evolution of IGR J17091.  I use 2 hardness ratios defined by \citeauthor{Altamirano_IGR_FH}: $H_{A1}$ and $H_{A2}$, corresponding to the ratios of the 3.5--6\,keV band against the 2--3.5\,keV band and the 9.7--16\,keV band against the 6--9.7\,keV band respectively.
\par When possible, if low-frequency peaks were present in the Lomb-Scargle spectrum of an observation, I used the position of the highest-amplitude peak to define a value for a period.  This period was then used to fold\index{Folding} the data to search for reccurent hysteretic\index{Hysteresis} patterns in the hardness-Intensity diagram\index{Hardness-intensity diagram} (hereafter HID$_1$, a plot of $L_T$ against $C_1$).  I found that quasi-periodic oscillations\index{Quasi-periodic oscillation} in the observations I used tended to show significant frequency shifts on timescales shorter than the length of the observations.  As such, I employed the variable-period folding algorithm outlined in Section \ref{sec:Flares} where appropriate.  For cases in which this algorithm was not appropriate, I considered small sections of each lightcurve, with a length equivalent to small number of periods, before performing folding.
\par Additionally, in observations which showed a pattern of high-amplitude X-ray flaring in $L_T$, I used my own algorithm to find individual flares\index{Flare} (this algorithm is described in Section \ref{sec:Flares}) and collect statistics on the amplitude, duration and profile of these events.
\par A list of all observations used in this study can be found in Appendix \ref{app:Obsids}.

\subsection{\textit{Swift}}

\par IGR J17091\index{IGR J17091-3624} was observed with \indexswift\textit{Swift}/XRT for a total of 172 pointed XRT observations between MJDs 55575 and 56600, corresponding to Target IDs 31921, 34543, 30967, 30973, 31920, 35096, 67137, 81917, 522245, 677582 and 677981.  These observations were interrupted during sun constraints centred on MJDs 55907 and 56272.  I created a long-term 0.3--10\,keV \indexxrt\textit{Swift/XRT} light curve, with one bin per pointed observation, using the online light-curve generator provided by the UK Swift Science Data Centre (UKSSDC; \citealp{Evans_Swift1}).  I have also created a long-term 15--50\,keV lightcurve using the publicly available \indexbat\textit{Swift/BAT} daily-averaged lightcurve\footnote{\url{http://swift.gsfc.nasa.gov/results/transients/weak/IGRJ17091-3624/}}.  These are shown in Figure \ref{fig:allmissions} Panels (b) and (c) respectively.

\subsection{\textit{INTEGRAL}}

\par Dr. Chris Boone (\textsf{C.B.}) and I analyse all available observations of IGR J17091 with \indexintegral\indexibis\textit{INTEGRAL}/IBIS \citep{Ubertini_IBIS} between MJD 55575--55625 where the source is less than 12 degrees from the centre of the field of view and where there is more than 1\,ks of good ISGRI time per 2\,ks Science Window. This corresponds to the spectrally hardest period of the 2011-2013 outburst. The filtering of observations results in a total of 188 Science Windows which were processed using the Offline Science Analysis (OSA) software version 10.2 following standard data reduction procedures\footnote{http://www.isdc.unige.ch/integral/analysis} in four energy bands (20--40, 40--100, 100--150, 150--300\,keV). These bands were selected as they are standard energy bands used in the surveys of \citet{Bird_Survey} and \citet{Bazzano_Survey} and allow comparison to these previous works. Images were created at the Science Window level, as well as a single mosaic of all Science Windows in each energy band.

\subsection{\textit{XMM-Newton}}
\label{sec:xmmdata}

\par \indexxmm\textit{XMM/Newton} observed IGR J17091 thrice during the period from 2011--2013 (represented by the blue arrows in Figure \ref{fig:allmissions}).  One of these observations (ObsID 0721200101) was made on 12 September 2013; I do not consider this observation further as IGR J17091 had returned to quiescence by this time \citep{Altamirano_Quiescence}.  The remaining two observations, corresponding to ObsIDs 0677980201 and 0700381301 respectively, were taken on March 27 2011 (MJD 55647) and September 29 2012 (MJD 56199).
\par During observation 0677980201, \indexepic\textit{EPIC-pn} was operating in burst mode and \textit{EPIC-MOS} was operating in timing mode.  Given the low efficiency of burst mode, I only consider data from \textit{EPIC-MOS} for this observation.  During observation 0700381301, \textit{EPIC-pn} was operating in timing mode, and thus I use data from \textit{EPIC-pn} for this observation.
\par I used the \textit{XMM-Newton} Science Analysis Software version 15.0.0 (\index{SAS@\texttt{SAS}}\texttt{SAS}, see \citealp{Ibarra_sas}) to extract calibrated event lists from \textit{EPIC} in both observations.  I used these to construct lightcurves to study the X-ray variability, following standard analysis threads\footnote{\url{http://www.cosmos.esa.int/web/xmm-newton/sas-threads}}.

\subsection{\textit{Chandra}}

\par \indexchandra\textit{Chandra} made 7 observations of IGR J17091 during the period 2011--2013.  Four of these observations were taken after IGR J17091 returned to quiescence, and I do not consider these further in this chapter.  The Chandra observations log is reported in Table \ref{tab:Chandra}. 

\begin{table}
\centering
\begin{tabular}{lllllll}
\hline
\hline
\scriptsize ObsID &\scriptsize  Instrument &\scriptsize Grating &\scriptsize Exposure (ks) &\scriptsize Mode &\scriptsize MJD\\
\hline
12505  	& \textit{HRC-I}    &   NONE      &    1.13      & $I$ & 55626\\
12405  	& \textit{ACIS-S} &   HETG     &    31.21     & $C$ & 55774\\
12406  	& \textit{ACIS-S} &   HETG     &    27.29     & $T$ & 55840\\
\hline
\hline
\end{tabular}
\caption[\textit{Chandra} observations log covering the three observations considered in Chapter \ref{ch:IGR}.]{\textit{Chandra} observations log covering the three observations considered in this chapter.  $I$ refers to Imaging mode, $C$ refers to CC33\_Graded mode and $T$ refers to Timed Exposure Faint mode.  HETG refers to the High Energy Transmission Grating.}
\label{tab:Chandra}
\end{table}

\par Dr. Margarita Pereyra (\textsf{M.P.}) analysed these data using \index{CIAO@\texttt{CIAO}}\texttt{CIAO} version 4.8 \citep{Fruscione_Ciao}, following the standard analysis threads. In order to apply the most recent calibration files (CALDB 4.7.0, \citealp{Graessle_ChaCALDB}), \textsf{M.P.} reprocessed the data from the three observations using the \texttt{chandra\_repro} script\footnote{See e.g. \url{http://cxc.harvard.edu/ciao/ahelp/chandra_repro.html}}, and used this to produce data products following standard procedures.
\par The first Chandra observation (ObsID 12505) of this source was made shortly after it went into outburst in February 2011. It was a 1\,ks observation performed to refine the position of the X-Ray source, using the High-Resolution Camera in Imaging mode \index{HRC}(HRC-I). \textsf{M.P.} created the 0.06--10\,kev light curve accounting for the Dead-Time\index{Dead-time} Factor (DTF), to correct the exposure time and count rate using the \texttt{dmextract} tool in the \texttt{CIAO} software.
\par Two additional observations (ObsIDs 12405 and 12406) were performed within 214 days of this first observation, using the High Energy Transmition Grating Spectrometer (HETGS) on board \textit{Chandra}. The incident X-Ray flux was dispersed onto \indexacis\textit{ACIS} using a narrow strip array configuration (ACIS-S). Continuous Clocking and Time Exposure modes were use in each observation respectively (see \citealp{King_IGRWinds} for further details). \textsf{M.P.} excluded any events below 0.4\,keV, since the grating efficiency is essentially zero below this energy. In the case of the ObsID 12405 observations \textsf{M.P.} also excluded the Flight Grade 66 events in the event file, as they were not appropriately graded. \textsf{M.P.} extracted the 0.5-10\,kev HEGTS light curves, excluding the zeroth-order flux, adopting standard procedures.

\subsection{\textit{Suzaku}}

\par \indexsuzaku\textit{Suzaku} observed IGR J17091 twice during the period 2011--2013; a 42.1\,ks observation on October 2--3, 2012 (MJD 56202--56203, ObsID: 407037010) and an 81.9\,ks observation on February 19--21, 2013 (MJD 56342--56344, ObsID: 407037020). \indexxis\textit{XIS} consists of four X-ray CCDs (\textit{XIS} 0, 1, 2 and 3), and all them except for XIS 2 were operating in the 1/4 window mode which has a minimum time resolution of 2 seconds.
\par Professor Kazutaka Yamaoka (\textsf{K.Y.}) analysed the \textit{Suzaku} data using \index{FTOOLS@\texttt{FTOOLS}}\index{HEASOFT@\texttt{HEASOFT}|see {\texttt{FTOOLS}}}\texttt{HEASOFT} 6.19 in the following standard procedures after reprocessing the data with \texttt{aepipeline} and the latest calibration database (version 20160607).  \textsf{K.Y.} extracted \textit{XIS} light curves in the 0.7--10 keV range, and subtracted background\index{Background subtraction} individually for XIS 0, 1 and 3 and then summed these to obtain the total background.  \textsf{K.Y.} created power density spectra\index{Fourier analysis} (PDS) using {\tt powspec} in the {\tt XRONOS} package.

\section{Results}
\label{sec:results}

\subsection{Outburst Evolution}

\label{sec:igrobevo}

\par The onset of the 2011-2013 outburst\index{Outburst} of \index{IGR J17091-3624}IGR J17091 can be seen in the \indexbat\textit{Swift/BAT} lightcurve (Figure \ref{fig:allmissions} Panel c).  In a 22 day period between MJDs 55584 and 55608, the 15--50\,keV intensity from IGR J17091 rose from $\sim9$\,mCrab to a peak of $\sim110$\,mCrab.  This onset rise in intensity can also be seen in 0.3--10\,keV \indexxrt\textit{Swift/XRT} data and 20--40\,keV \indexibis\textit{INTEGRAL/ISGRI} data.
\par After peak intensity, the 15--50\,keV flux (\textit{Swift/BAT}) began to steadily decrease, until returning to a level of $\sim$20\,mCrab by MJD 55633.  A similar decrease in flux can be seen in the data obtained by \textit{INTEGRAL} at this time (Figure \ref{fig:allmissions} Panel (d).  However, there was no corresponding fall in the flux at lower energies; both the long-term 2--16\,keV \indexpca\rxte\ /PCA data and \textit{Swift/XRT} data (Panels a and b respectively) show relatively constant fluxes of 45\,mCrab between MJDs 55608 and 55633.
\par The significant decrease in high-energy flux during this time corresponds to IGR J17091 transitioning from a hard state\index{Low/Hard state} to a soft/intermediate state\index{High/Soft state} \citep{Pahari_RhoDiff}.  This transition coincides with a radio flare reported by \citet{Rodriguez_D} which was observed by the Australian Telescope Compact Array (ATCA\index{ATCA}).
\par \citealp{Altamirano_10Hz} first reported a 10\,mHz QPO\index{Quasi-periodic oscillation} in \rxte\ data on MJD 55634 , evolving into `Heartbeat-like'\indexrho\ flaring\index{Flare} by MJD 55639 \citep{Altamirano_Discovery}.  Between MJDs 55634 and 55879, the global \indexpca\rxte\ /PCA lightcurve shows large fluctuations in intensity on timescales of days to weeks, ranging from a minimum of $\sim20$\,mCrab on MJD 55731 to a maximum of $\sim66$\,mCrab on MJD 55756.  The \indexxrt\textit{Swift/XRT} lightcurve shows fluctuations that mirror those seen by \rxte\ during this period, but the amplitude of the fluctuations is significantly reduced.
\par \indexxrt\textit{Swift/XRT} was unable to observe again until MJD 55952.  Between this date and MJD 55989, \textit{Swift/XRT} observed a gradual decrease in intensity corresponding to a return to the low/hard state\index{Low/Hard state} \citep{Drave_Return}.
\par Between MJD 55989 and the end of the outburst on MJD 56445, there are secondary peaks in the \indexxrt\textit{Swift/XRT}, \indexbat\textit{Swift/BAT} and \indexibis\textit{INTEGRAL/ISGRI} lightcurves that evolve over timescales of $\lesssim100$ days.  Similar humps have been seen before in lightcurves from other objects, for example the black hole candidate XTE J1650-500 \citep{Tomsick_MiniOutbursts} and the neutron stars SAX J1808.4-3658 \citep{Wijnands_1808} and SAX J1750.8-2900 \citep{Allen_1750}.  These humps are referred to as `re-flares'\index{Re-flare} (also as `rebrightenings',  `echo-outbursts', `mini-outbursts' or a `flaring tail', e.g. \citealp{Patruno_Reflares2}).  I identify a total of 3 apparent re-flares\index{Re-flare} in the \indexbat\textit{Swift/BAT} data, centred approximately at MJDs 56100, 56220 and 56375.
\par The observation with \indexepic\textit{XMM-Newton/EPIC-pn} on MJD 56547 (12 September 2013) recorded a rate of 0.019 cts s$^{-1}$.  An observation with \textit{EPIC-pn} in 2007, while IGR J17091\index{IGR J17091-3624} was in quiescence\index{Quiescence} \citep{Wijnands_Quiescence}, detected a similar count rate of 0.020 cts s$^{-1}$.  Therefore I define MJD 56547 as the upper limit on the endpoint of the 2011-2013 outburst.  As such the outburst, as defined here, lasted for $\lesssim$952 days.
\par After the end of the 2011-2013 outburst, IGR J17091 remained in quiescence\index{Quiescence} until the start of a new outburst\index{Outburst} around MJD 57444 (26 February 2016, \citealp{Miller_2016Outburst}).

\subsection{\rxte}

\label{sec:IGRclassesintro}

\par Using the \rxte\indexrxte\
 data products described in Section \ref{sec:dex}, I assigned a model-independent variability class\index{Variability class} to each of the 243 \rxte\textit{/PCA}\indexpca\ orbits during which IGR J17091 was observed\index{IGR J17091-3624}.  To avoid bias, this was done without reference to the classes defined by \citet{Belloni_GRS_MI} to describe the behaviour of GRS 1915\index{GRS 1915+105}.
\par Classes were initially assigned based on by-eye analysis of lightcurve\index{Lightcurve} profiles, count rate, mean fractional RMS\index{RMS} \citep{Vaughan_RMS}, Fourier power spectra and Lomb-scargle periodograms,\index{Fourier analysis}{Lomg-Scargle periodogram} and hardness-intensity diagrams\index{Hardness-intensity diagram}.  For observations with significant quasi-periodic variability at a frequency lower than $\sim1$\,Hz, I also attempted to fold\index{Folding} lightcurves to analyse count rate and colour as a function of phase.  When flares\index{Flare} were present in the lightcurve, I used my algorithm (described in Section \ref{sec:Flares}) to sample the distribution of parameters such as peak flare count rate, flare rise time and flare fall time.  All parameters were normalised per active PCU, and fractional RMS\index{RMS} values were taken from 2--60\,keV lightcurves binned to 0.5\,s.  I identify nine distinct classes, labelled I to IX; I describe these in the following sections.
\par Although the criteria for assigning each class to an observation was different, a number of criteria were given the most weight.  In particular, the detection, $q$-value\indexq\ and peak frequency of a QPO\index{Quasi-periodic oscillation} in the range 2\,Hz--10\,Hz were used as criteria for all classes, as well as the presence or absence of high-amplitude quasi-periodic flaring with a frequency between 0.01--1\,Hz.  The folded\index{Folding} profile of these flares, as well as the presence of associated harmonics\index{Harmonic}, were also used as classification diagnostics in observations.  Additionally, the presence or absence of low count-rate 'dips'\index{Dip} in a lightcurve was used as a criterion for Classes VI\indexvi, VIII\indexviii\ and IX\indexix.  Detailed criteria for each individual class are given below in Sections \ref{sec:ClassI} to \ref{sec:ClassIX}.  As each observation lasted less than $\lesssim3$\,ks, significantly shorter than the timescale over which IGR J17091-3624 evolved between classes, a single class could be assigned to all observations\footnote{See however Figure \ref{fig:HybridClasses} for an example lightcurve of an observation which appeared to capture a transition between two classes.}.
\par For hardness-intensity diagrams\index{Hardness-intensity diagram}, I describe looping behaviour\index{Hysteresis}\index{Loops|see {Hysteresis}} with the terms `clockwise' and `anticlockwise'; in all cases, these terms refer to the direction of a loop plotted in a hardness-intensity diagram with colour on the $x$-axis and intensity on the $y$-axis.  I did not study these hysteretic loops until after I had established my set of variability classes, and hence the presence or direction of a loop was not used as a diagnostic feature to assign a class to an observation.
\par In Appendix \ref{app:Obsids}, I present a list of all orbits used in the study along with the variability classes I assigned to them.
\par In Figure \ref{fig:WhereCls}, I show global 2--16\,keV\index{Lightcurve} lightcurves of IGR J17091 during the 2011-2013 outburst.  In each panel, all observations of a given class\index{Variability class} are highlighted in red.  A characteristic lightcurve is also presented for each class.  In Figure \ref{fig:IIIisHarder} panel (a), I show a plot of average hardness $H_{A2}$ against $H_{A1}$\index{Colour} for each observation, showing the long-term hysteresis\index{Hysteresis} of the object in colour-colour space\index{Colour-colour diagram}.  Again, observations belonging to each variability class are highlighted.  In Figure \ref{fig:IIIisHarder} panels (b) and (c), I show global hardness-intensity diagrams\index{Hardness-intensity diagram} for $H_{A1}$ and $H_{A2}$ respectively.
\par In Figure \ref{fig:IIIisHarder} Panel (a), we see that IGR J17091-3624\index{IGR J17091-3624} traces a two branched pattern in colour-colour\index{Colour-colour diagram} space corresponding to a branch which is soft ($\sim0.9$) in $H_{A1}$ and variable in $H_{A2}$ and a branch which is soft ($\sim0.5$) in $H_{A2}$ and variable in $H_{A1}$.  The `soft' HID shown in Figure \ref{fig:IIIisHarder} Panel (b) is dominated by a branch with a wide spread in $H_{A1}$ and intensities between $\sim40\mbox{--}60$\,mCrab.  A second branch exists at lower intensities, and shows an anticorrelation between intensity and $H_{A1}$.  Finally, the `hard' HID shown in Figure \ref{fig:IIIisHarder} Panel (c) shows an obvious anticorrelation between $H_{A2}$ and intensity, but there is also a secondary branch between $H_{A2}\approx 0.7\mbox{--}0.9$ at a constant intensity of $\sim40$\,mCrab.

\begin{figure}
    \includegraphics[width=\columnwidth, trim = {1.3cm 2.0cm 1.8cm 1.8cm},clip]{images/all_lc.eps}
    \captionsetup{singlelinecheck=off}
    \caption[Lightcurves of IGR J17091-3524 during the 2011-2013 outburst, showing when each of variability classes I-IX were observed.]{Global 2--3.5\,keV Lightcurves of IGR J17091-3524\index{IGR J17091-3624} during the 2011-2013 outburst, with each point corresponding to the mean Crab-normalised\index{Crab nebula} count rate of a single \indexrxte\rxte\ observation of the object (in turn corresponding to between 0.4 and 3.6 ks of data).  In each lightcurve, every observation identified as belonging to a particular class\index{Variability class} (indicated on the plot) is highlighted.  These are presented along with a characteristic lightcurve (inset) from an observation belonging to the relevant class.  Each lightcurve is 250\,s in length, and has a $y$-scale from 0 to 250\spcu .  Data taken from \citealt{Altamirano_IGR_FH}.}
   \label{fig:WhereCls}
\end{figure}

\begin{figure}
\centering
\subfloat[\textit{Colour-Colour Diagram}]{\includegraphics[width=0.8\columnwidth, trim = 0mm 0mm 0mm 8mm,clip]{images/all_ccd.eps}}\\
\subfloat[\textit{`Soft' ($H_{A1}$) Hardness-Intensity Diagram}]{\includegraphics[width=0.8\columnwidth, trim = 0mm 0mm 0mm 8mm,clip]{images/all_shid_colorful.eps}}\\
\subfloat[\textit{`Hard' ($H_{A2}$) Hardness-Intensity Diagram}]{\includegraphics[width=0.8\columnwidth, trim = 0mm 0mm 0mm 8mm,clip]{images/all_hhid_colorful.eps}}\\
\captionsetup{singlelinecheck=off}
\caption[Hardness-Intensity diagrams of IGR J17091-3524 during the 2011-2013 outburst, showing when each of variability classes I-IX were observed.]{A global colour-colour diagram\index{Colour-colour diagram} (a), `soft' hardness-intensity diagram\index{Hardness-intensity diagram} (b) and `hard' hardness-intensity diagram (c) of the 2011-2013 outburst of IGR J17091, using the colours $H_{A1}$ and $H_{A2}$ defined previously.  Observations belonging to different classes have been highlighted in different colours.  Data taken from \citealt{Altamirano_IGR_FH}.}
\label{fig:IIIisHarder}
\end{figure}

\par For characteristic count rates and colours in each class, I quote the upper and lower quartile values \citep{Kenney_Quartile} instead of the mean.  This is due to the presence of high-amplitude but short-lived flares in many of the classes I describe.  Using the upper and lower quartiles as my measure of average and distribution means that my values will be less susceptible to outlier values of count rate and colour present in these flares.  All count rates have been background corrected\index{Background subtraction} (see Section \ref{sec:XTEDA}).
\par I have obtained mean values for these count rate quartiles, as well as values for colour\index{Colour} $C_1$ and fractional RMS\index{RMS}, by calculating these values individually for each orbit.  Histograms were then constructed from these datasets for each class, such that the mean and standard deviation of these values could be measured for each class.  These values are presented in Table \ref{tab:basicparams}.
\par I describe QPOs\index{Quasi-periodic oscillation} in terms of their $q$-value\indexq; a measure of coherence defined by the ratio of peak frequency and full-width half-maximum of each QPO.  I collected these values by fitting my power spectra with Lorentzians.

\begin{table}
\centering
\begin{tabular}{rllll} % four columns, alignment for each
\hline
\hline
\scriptsize Class &\scriptsize LQ Rate &\scriptsize  UQ Rate &\scriptsize Frac. RMS &\scriptsize Median C$_1$\\
\scriptsize &\scriptsize (cts s$^{-1}$) &\scriptsize (cts s$^{-1}$) & & \\
\hline
I\indexi&84--108&106--132&0.13--0.19&0.4--0.68\\
II\indexii&43--57&59--71&0.15--0.23&0.4--0.68\\
III\indexiii&64--84&80--110&0.17-0.23&0.35--0.45\\
IV\indexiv&63--81&92--122&0.27--0.37&0.32--0.4\\
V\indexv&49--67&88--134&0.44--0.54&0.28--0.46\\
VI\indexvi&64--98&111--155&0.29--0.47&0.33--0.61\\
VII\indexvii&65--79&128--140&0.45--0.57&0.32--0.42\\
VIII\indexviii&62--88&142--178&0.42--0.52&0.36--0.49\\
IX\indexix&87--111&114--144&0.16--0.24&0.42-0.6\\
\hline
\hline
\end{tabular}
\caption[A number of statistics averaged across all observations belonging to each IGR J17091 variability class.]{Lower and upper quartile count rates, fractional RMS\index{RMS} and median colour averaged across all observations belonging to each class\index{Variability class}.  Count rates and fractional RMS are taken from the full energy range of \indexpca\rxte\textit{/PCA}, and fractional RMS values are 2--60\,keV taken from lightcurves binned to 0.5\,s.  Count rates are normalised for the number of PCUs active during each observation.  All values are quoted as $1\sigma$ ranges.}
\label{tab:basicparams}
\end{table}

\par For each class\index{Variability class}, I present three standard data products; a 500\,s lightcurve\index{Lightcurve}, a variable-length lightcurve where the length has been selected to best display the variability associated with the class and a Fourier PDS\index{Fourier analysis}.  Unless otherwise stated in the figure caption, the 500\,s lightcurve and the Fourier PDS are presented at the same scale for all classes.  In Table \ref{tab:CPopD} I present a tally of the number of times I assigned each Variability Class to an \rxte\ orbit.

\begin{table}
\centering
\begin{tabular}{llll}
\hline
\hline
\scriptsize Class &\scriptsize  Orbits &\scriptsize Total Time (s) &\scriptsize Fraction \\
\hline
I\indexi\ & 31 &  69569 & 14.8\%\\
II\indexii\ & 26 &  50875 & 10.8\%\\
III\indexiii\ & 14 &  26228 & 5.6\%\\
IV\indexiv\ & 31 &  69926 & 14.9\%\\
V\indexv\ & 35 &  72044 & 15.3\%\\
VI\indexvi\ & 29 &  54171 & 11.5\%\\
VII\indexvii\ & 11 &  19241 & 4.1\%\\
VIII\indexviii\ & 16 &  26553 & 5.7\%\\
IX\indexix\ & 50 &  81037 & 17.3\%\\
\hline
\hline
\end{tabular}
\caption[A tally of the number of times I assigned each of my nine Variability Classes to an \rxte\ orbit observing IGR J17091.]{A tally of the number of times I assigned each of my nine Variability Classes\index{Variability class} to an \rxte\ orbit.  I have also calculated the amount of observation time corresponding to each class, and thus inferred the fraction of the time that IGR J17091\index{IGR J17091-3624} spent in each class.  Note: the values in the Total Time column assume that each orbit only corresponds to a single variability Class.}
\label{tab:CPopD}
\end{table}

\subsubsection{Class I --  Figure \ref{fig:Bmulti}}
\label{sec:ClassI}

\begin{figure}
    \includegraphics[width=0.8\columnwidth, trim = 0.6cm 0 3.9cm 0]{images/Bmulti.png}
    \captionsetup{singlelinecheck=off}
    \caption[Characteristic lightcurves and a power spectrum of Type I variability.]{Plots of the Class I\indexi\ observation 96420-01-01-00, orbit 0.  \textit{Top-left}: 1000\,s lightcurve\index{Lightcurve} binned on 2 seconds to show lightcurve evolution.  \textit{Top-right}: Fourier Power Density Spectrum\index{Fourier analysis}.  \textit{Bottom}: 500\,s lightcurve binned on 2 seconds.}
   \label{fig:Bmulti}
\end{figure}

In the 2\,s binned lightcurve\index{Lightcurve} of a Class I\indexi\ observation, there is no structured second-to-minute scale variability.  The Fourier PDS\index{Fourier analysis} of all observations in this class show broad band noise between $\sim1$--$10$\,Hz, as well as a weak QPO\index{Quasi-periodic oscillation} (with a $q$-value\indexq\ of $\sim5$) which peaks at around 5\,Hz.

\subsubsection{Class II -- Figure \ref{fig:Emulti}}

\begin{figure}
    \includegraphics[width=0.8\columnwidth, trim = 0.6cm 0 3.9cm 0]{images/Emulti.png}
    \captionsetup{singlelinecheck=off}
    \caption[Characteristic lightcurves and a power spectrum of Type II variability.]{Plots of the Class II\indexii\ observation 96420-01-11-00, orbit 0.  \textit{Top-left}:  1000\,s lightcurve\index{Lightcurve} binned on 2 seconds to show lightcurve evolution.  \textit{Top-right}: Fourier Power Density Spectrum\index{Fourier analysis}.  \textit{Bottom}: Lightcurve binned on 2 seconds.}
   \label{fig:Emulti}
\end{figure}

\par Class II\indexii\ observations are a factor of $\sim2$ fainter in the $L_T$ band than Class I\indexi\ observations.  They also occupy a different branch in a plot of hardness $H_{A2}$\index{Hardness-intensity diagram} against intensity (see Figure \ref{fig:IIIisHarder}, panel c).  The PDS\index{Fourier analysis} shows no significant broad band noise above $\sim1$\,Hz unlike that which is seen in Class I.  The $\sim$5\,Hz QPO\index{Quasi-periodic oscillation} seen in Class I is absent in Class II.

\subsubsection{Class III -- Figure \ref{fig:Gmulti}}
\label{sec:classIII}

\begin{figure}
    \includegraphics[width=0.8\columnwidth, trim = 0.6cm 0 3.9cm 0]{images/Gmulti.png}
    \captionsetup{singlelinecheck=off}
    \caption[Characteristic lightcurves and a power spectrum of Type III variability.]{Plots of the Class III\indexiii\ observation 96420-01-04-01, orbit 0.  \textit{Top-left}: 1000\,s lightcurve\index{Lightcurve} binned on 2 seconds to show lightcurve evolution.  \textit{Top-right}: Fourier Power Density Spectrum\index{Fourier analysis}.  \textit{Bottom}: Lightcurve binned on 2 seconds.  Note that, to emphasise the behaviour of the lightcurve in this class, I have magnified the 500\,s lightcurve y-scale by a factor of 2 compared with the lightcurves presented for other classes.}
   \label{fig:Gmulti}
\end{figure}

\par Unlike Classes I \& II, Class III\indexiii\ lightcurves show structured flaring\index{Flare}, with a peak-to-peak recurrence time\index{Recurrence time} of $42$--$80$\,s.  Most flares consist of a steady $\sim60$\,s rise in count rate and then an additional and sudden rise to a peak count rate at $\gtrsim200$\spcu which lasts for $\lesssim$0.5\,s before returning to continuum level (I have magnified the y-scaling in the lightcurve of Figure \ref{fig:Gmulti} to emphasise this behaviour). This sudden rise is not present in every flare; in some observations it is absent from every flare feature.  No 5\,Hz QPO\index{Quasi-periodic oscillation} is present in the PDS\index{Fourier analysis} and there is no significant variability in the range between $\sim1\mbox{--}10$\,Hz.

\par As this class has a well-defined periodicity, I folded\index{Folding} data in each observation to improve statistics using the best-fit period obtained from generalised Lomb-Scargle Periodogram Analysis; I show a representative Lomb-Scargle periodogram\index{Lomb-Scargle periodogram} in Figure \ref{fig:IIILS}.  I find an anticlockwise hysteretic\index{Hysteresis} loop in the folded HID$_1$\index{Hardness-intensity diagram} of all 15 Class III orbits.  In Figure \ref{fig:LoopIII} I show an example of one of these loops.

\begin{figure}
    \includegraphics[width=\columnwidth, trim = 0mm 0mm 0mm 0mm]{images/LSVIII.eps}
    \captionsetup{singlelinecheck=off}
    \caption[The Lomb-Scargle periodogram of Class III observation 96420-01-19-01]{The Lomb-Scargle periodogram\index{Lomb-Scargle periodogram} of Class III\indexiii\ observation 96420-01-19-01, orbit 0, with significance levels of 1, 2 and 3$\sigma$ plotted.  The peak at 0.31\,Hz was used to define a QPO\index{Quasi-periodic oscillation} frequency when folding the data from this observation.}
   \label{fig:IIILS}
\end{figure}

\begin{figure}
    \includegraphics[width=\columnwidth, trim = 0mm 0mm 0mm 0mm]{images/Gloop.png}
    \captionsetup{singlelinecheck=off}
    \caption[A hardness-intensity diagram of the Class III observation 96420-01-04-01.]{The hardness-intensity diagram (HID$_1$)\index{Hardness-intensity diagram} of the Class III\indexiii\ observation 96420-01-04-01, orbit 0.  The data have been folded\index{Folding} over a period of 79.61 s, corresponding to the peak frequency in the Lomb-Scargle periodogram\index{Lomb-Scargle periodogram} of this observation.  Inset is the folded lightcurve\index{Lightcurve} of the same data.}
   \label{fig:LoopIII}
\end{figure}

\subsubsection{Class IV -- Figure \ref{fig:Jmulti}}
\label{sec:classIV}

\begin{figure}
    \includegraphics[width=0.8\columnwidth, trim = 0.6cm 0 3.9cm 0]{images/Jmulti.png}\\
    \captionsetup{singlelinecheck=off}
    \caption[Characteristic lightcurves and a power spectrum of Type IV variability.]{Plots of the Class IV\indexiv\ observation 96420-01-05-00, orbit 0.  \textit{Top-left}: 1000\,s lightcurve\index{Lightcurve} binned on 2 seconds to show lightcurve evolution.  \textit{Top-right}: Fourier Power Density Spectrum\index{Fourier analysis}.  \textit{Bottom}: Lightcurve binned on 0.5 seconds.}
   \label{fig:Jmulti}
\end{figure}

\par The lightcurves\index{Lightcurve} in this class\indexiv\ show regular variability\index{Variability} with a peak-to-peak recurrence time\index{Recurrence time} of $25$--$39$\,s.  I performed peak analysis (see Section \ref{sec:Flares}) on observations belonging to this class, finding that each flare\index{Flare} has a rise time with lower and upper quartile values of $19.5$ and $33.5$ s, a fall time with lower and upper quartile values of $4.6$ and $13.5$\,s and a peak count rate of $159$--$241$\spcu\ .  There are no significant QPOs\index{Quasi-periodic oscillation} in the Fourier PDS above $\sim1$\,Hz.
\par I folded\index{Folding} individual Class IV\indexiv\ lightcurves and found anticlockwise hysteretic\index{Hysteresis} loops in the HID$_1$\index{Hardness-intensity diagram} of 14 out of 30 Class IV observations.  In the top panel of Figure \ref{fig:LoopIV} I show an example of one of these loops.  However, I also find clockwise hysteretic loops in 6 Class IV observations, and in 10 orbits the data did not allow us to ascertain the presence of a loop.  I provide an example of both of these in the lower panels of Figure \ref{fig:LoopIV}.  I note that the structure of clockwise loops are more complex than anticlockwise loops in Class IV, consisting of several lobes\footnote{In HIDs with multiple lobes, the loop direction I assign to the observation corresponds to the direction of the largest lobe.} rather than a single loop (Figure \ref{fig:LoopIV}, bottom-left).

\begin{figure}
    \includegraphics[width=\columnwidth, trim = 0mm 0mm 0mm 0mm]{images/Jloop.png}\\
    \includegraphics[width=0.5\columnwidth, trim = 0mm 0mm 0mm 0mm]{images/Jloop4.png}\includegraphics[width=0.5\columnwidth, trim = 0mm 0mm 0mm 0mm]{images/Jloop3.png}
    \captionsetup{singlelinecheck=off}
    \caption[The hardness-intensity diagram of the Class IV observation 96420-01-05-00, showing an anticlockwise loop.]{\textit{Top}: The hardness-intensity diagram\index{Hardness-intensity diagram} (HID$_1$) of the Class IV\indexiv\ observation 96420-01-05-00, orbit 0 showing an anticlockwise loop\index{Hysteresis}.  The data have been folded\index{Folding} over a variable period found with the algorithm described in Section \ref{sec:Flares}.  Inset is the folded lightcurve\index{Lightcurve} of the same data.  \textit{Bottom Left}: The hardness-intensity diagram of Class IV observations 96420-01-24-02 orbit 0, an example of a clockwise loop.  \textit{Bottom Right}: The hardness-intensity diagram of Class IV observation 96420-01-06-00 orbit 0, in which I was unable to ascertain the presence of a loop.}
   \label{fig:LoopIV}
\end{figure}

\par Compared with Class III\indexiii, the oscillations in Class IV\indexiv\ occur with a significantly lower period, with a mean peak-to-peak recurrence time\index{Recurrence time} of $\sim30$\,s compared to $\sim60$\,s in Class III.
\par In Figure \ref{fig:IIIisHarder} I show that Classes III\indexiii\ and IV\indexiv\ can also be distinguished by average hardness\index{Colour}, as Class III tends to have a greater value of $H_{A2}$ than Class IV.

\subsubsection{Class V -- Figure \ref{fig:Kmulti}}
\label{sec:classV}

\begin{figure}
    \includegraphics[width=0.8\columnwidth, trim = 0.6cm 0 3.9cm 0]{images/Kmulti.png}
    \captionsetup{singlelinecheck=off}
    \caption[Characteristic lightcurves and a power spectrum of Class V variability.]{Plots of the Class V\indexv\ observation 96420-01-06-03, orbit 0.  \textit{Top-left}: 750\,s lightcurve\index{Lightcurve} binned on 2 seconds to show lightcurve evolution.  \textit{Top-right}: Fourier Power Density Spectrum\index{Fourier analysis}. \textit{Bottom}: Lightcurve binned on 0.5 seconds.}
   \label{fig:Kmulti}
\end{figure}

\par The lightcurves\index{Lightcurve} in this class\indexv, like in Classes III\indexiii\ and IV\indexiv, show flaring\index{Flare} behaviour, with flares separated by a few tens of seconds.  At higher frequencies, the PDS\index{Fourier analysis} shows a prominent QPO\index{Quasi-periodic oscillation} centred at $\sim4$\,Hz with as $q$-value\indexq\ of $\sim3$.  There is also significant broad band noise between $\sim0.1$--$1$\,Hz
\par In Figure \ref{fig:id_flares_V} I show that the flaring in this class is more complex than that seen in Classes III and IV.  Class V lightcurves consist of short strongly peaked symmetrical flares\index{Flare} (hereafter Type $V_1$) and a longer more complex type of flare (hereafter Type $V_2$).  The Type $V_2$ flare consists of a fast rise to a local maximum in count rate, followed by a $\sim10$\,s period in which this count rate gradually reduces by $\sim50\%$ and then a much faster peak with a maximum count rate between 1 and 2 times that of the initial peak.  In both types of flare, I find that the increase in count rate corresponds with an increase in soft colour\index{Colour}.  The two-population nature\index{Population study} of flares in Class V can also clearly be seen in Figure \ref{fig:two_popV}, where I show a two-dimensional histogram of flare peak count rate against flare duration.
\par I folded\index{Folding} all individual Class V\indexv\ lightcurves, in each case cropping out periods of $V_2$ flaring.  I find clockwise hysteretic\index{Hysteresis} loops in the HID$_1$\index{Hardness-intensity diagram} of 30 out of 33 Class V observations, suggesting a lag\index{Hard lag} in the aforementioned relation between count rate and soft colour.  In the upper panel Figure \ref{fig:LoopV} I present an example of one of these loops.  In one observation however, I found an anticlockwise loop in the HID$_1$ (shown in Figure \ref{fig:LoopV} lower-left panel).  I was unable to ascertain the presence of loops in the remaining 2 orbits; for the sake of completeness, I show one of these in the lower-right panel of Figure \ref{fig:LoopV}.

\begin{figure}
    \includegraphics[width=\columnwidth, trim = 0mm 0mm 0mm 0mm]{images/KBurstTypes.png}
    \captionsetup{singlelinecheck=off}
    \caption[A portion of the lightcurve of observation 96420-01-06-03 showing Type $V_1$ flares and Type $V_2$ flares.]{A portion of the lightcurve\index{Lightcurve} of observation 96420-01-06-03, orbit 0, showing Type $V_1$\indexv\index{Flare} flares (highlighted in cyan) and Type $V_2$ flares (highlighted in red).}
   \label{fig:id_flares_V}
\end{figure}

\begin{figure}
    \includegraphics[width=\columnwidth, trim = 0mm 0mm 0mm 0mm]{images/KBurst.png}
    \captionsetup{singlelinecheck=off}
    \caption[Every flare in all observations identified as Class V, plotted in a two-dimensional histogram of flare peak count rate against flare duration to show the two-population nature of these events.]{Every flare\index{Flare} in all observations identified as Class V\indexv, plotted in a two-dimensional histogram of flare peak count rate against flare duration to show the two-population\index{Population study} nature of these events.}
   \label{fig:two_popV}
\end{figure}

\begin{figure}
    \includegraphics[width=\columnwidth, trim = 0mm 0mm 0mm 0mm]{images/Kloop.png}\\
    \includegraphics[width=0.5\columnwidth, trim = 0mm 0mm 0mm 0mm]{images/Kloop2.png}\includegraphics[width=0.5\columnwidth, trim = 0mm 0mm 0mm 0mm]{images/Kloop3.png}
    \captionsetup{singlelinecheck=off}
    \caption[The hardness-intensity diagram of a type $V_1$ flaring period in Class V observation 96420-01-07-00 showing a clockwise loop.]{\textit{Top}: The hardness-intensity diagram\index{Hardness-intensity diagram} (HID$_1$) of a type $V_1$\indexv\ flaring\index{Flare} period in Class V observation 96420-01-07-00, orbit 0 showing a clockwise loop\index{Hysteresis}.  The data have been folded\index{Folding} over a variable period found with the algorithm described in Section \ref{sec:Flares}.  Inset is the folded lightcurve\index{Lightcurve} of the same data. \textit{Bottom Left}: The hardness-intensity diagram of Class V observation 96420-01-25-05 orbit 0, an example of an anticlockwise loop.  \textit{Bottom Right}: The hardness-intensity diagram of Class V observation 96420-01-25-06 orbit 0, in which I was unable to ascertain the presence of a loop.}
   \label{fig:LoopV}
\end{figure}

\subsubsection{Class VI -- Figure \ref{fig:Lmulti}}

\begin{figure}
    \includegraphics[width=0.8\columnwidth, trim = 0.6cm 0 3.9cm 0]{images/Lmulti.png}
    \captionsetup{singlelinecheck=off}
    \caption[Characteristic lightcurves and a power spectrum of Type VI variability.]{Plots of the Class VI observation 96420-01-09-00, orbit 0.  \textit{Top-left}: 750\,s lightcurve\index{Lightcurve} binned on 2 seconds to show lightcurve evolution.  \textit{Top-right}: Fourier Power Density Spectrum\index{Fourier analysis}.  \textit{Bottom}: Lightcurve binned on 1 second.}
   \label{fig:Lmulti}
\end{figure}

\par The lightcurves\index{Lightcurve} of observations of this class\indexvi\ show large dips\index{Dip} in count rate; this can be seen in Figure \ref{fig:Lmulti} at, for example, $t\approx125$--$150$\,s .  These dips vary widely in duration, from $\sim5$ to $\sim50$ seconds, and the count rate in both $L_A$ and $L_B$ fall to a level consistent with background.  The dips' rise and fall times are fast, both lasting no longer than a second.  They do not appear to occur with any regular periodicity.
\par Aside from the dips, Class VI\indexvi\ observations show other structures in their lightcurves.  Large fluctuations in count rate, by factors of $\lesssim3$, occur on timescales of $\sim1\mbox{--}5$ s; no periodicity in these oscillations could be found.  This behaviour is reflected in the PDS\index{Fourier analysis}, which shows high-amplitude broad band noise below $\sim0.5$\,Hz with RMS-normalized power\index{RMS normalisation} \citep{Belloni_RMSNorm} of up to $\sim1.1 $\,Hz$^{-1}$.  As can be seen in Figure \ref{fig:Lmulti}, this feature takes the form of a broad shoulder of noise which shows either a weak peak or no clear peak at all.  The $\sim5$\,Hz QPO\index{Quasi-periodic oscillation} seen in the PDS of other classes is not present in Class VI observations.
\par I attempted to fold\index{Folding} all individual Class VI\indexvi\ lightcurves, ignoring the sections of data corresponding to the large count rate dips described above.  In general, folding lightcurves belonging to this class is difficult; many orbits showed low-amplitude oscillations which were difficult to fold using my flare-finding algorithm (see Section \ref{sec:Flares}), while many others only showed oscillatory behaviour for a small number of periods between each pair of dips.  As such, I only succesfully folded 23 of the 40 Class VI orbits.  Of these, 19 showed clockwise loops\index{Hysteresis} in the HID$_1$\index{Hardness-intensity diagram} (top panel, Figure \ref{fig:LoopVI}), 3 showed anticlockwise loops (bottom-left panel, Figure \ref{fig:LoopVI}).  In the remaining 1 observation, the data did not allow us to ascertain the presence of loops (bottom-right panel, Figure \ref{fig:LoopVI}).
\par Like in Class VI, I note that the clockwise loops in Class VI appear more complex than clockwise loops.  Again, the clockwise loop shown in Figure \ref{fig:LoopVI} appears to have a 2-lobe structure; this is repeated in all clockwise loops found in this class.

\begin{figure}
    \includegraphics[width=\columnwidth, trim = 0mm 0mm 0mm 0mm]{images/Lloop2.png}\\
    \includegraphics[width=0.5\columnwidth, trim = 0mm 0mm 0mm 0mm]{images/Lloop.png}\includegraphics[width=0.5\columnwidth, trim = 0mm 0mm 0mm 0mm]{images/Lloop4.png}
    \captionsetup{singlelinecheck=off}
    \caption[The hardness-intensity diagram of the Class VI observation 96420-01-30-03, showing a clockwise loop.]{\textit{Top}: The hardness-intensity diagram\index{Hardness-intensity diagram} (HID$_1$) of the Class VI\indexvi\ observation 96420-01-30-03, orbit 0 showing a clockwise loop\index{Hysteresis}.  The data have been folded\index{Folding} over a variable period found with the algorithm described in Section \ref{sec:Flares}.  Inset is the folded lightcurve\index{Lightcurve} of the same data. \textit{Bottom Left}: The hardness-intensity diagram of Class VI observation 96420-01-30-04 orbit 0, an example of an anticlockwise loop.  \textit{Bottom Right}: The hardness-intensity diagram of Class VI observation 96420-01-09-03 orbit 0, in which I was unable to ascertain the presence of a loop.}
   \label{fig:LoopVI}
\end{figure}

\subsubsection{Class VII -- Figure \ref{fig:Nmulti}}

\begin{figure}
    \includegraphics[width=0.8\columnwidth, trim = 0.6cm 0 3.9cm 0]{images/Nmulti.png}
    \captionsetup{singlelinecheck=off}
    \caption[Characteristic lightcurves and a power spectrum of Type VII variability.]{Plots of the Class VII\indexvii\ observation 96420-01-18-05, orbit 0.  \textit{Top-left}: 750\,s lightcurve\index{Lightcurve} binned on 2 seconds to show lightcurve evolution.  \textit{Top-right}: Fourier Power Density Spectrum\index{Fourier analysis}.  \textit{Bottom}: Lightcurve binned on 0.5 seconds.}
   \label{fig:Nmulti}
\end{figure}

\par Class VII\indexvii\ shows high-amplitude flaring\index{Flare} behaviour with a peak-to-peak recurrence time\index{Recurrence time} of $6$--$12$\,s.  In Figure \ref{fig:spect} I show a dynamical Lomb-Scargle spectrogram\index{Lomb-Scargle periodogram} of a Class VII observation, showing that the fast flaring behaviour has a frequency which moves substantially over time.  This in turn accounts for the large spread in the value of the flare peak-to-peak recurrence time.
\par In Figure \ref{fig:spect} I show that the peak frequency of the QPO\index{Quasi-periodic oscillation} also varies in a structured way.  I also suggest that the variabilitity of the frequency is itself a QPO with a period of $\sim150$\,s.
\par At higher frequencies, the PDS\index{Fourier analysis} shows a weak QPO\index{Quasi-periodic oscillation} centred at $\sim8$\,Hz, with a $q$-value\indexq\ of $\sim2$.
\par I used my flare-finding algorithm (see Section \ref{sec:Flares}) to perform variable-frequency folding\index{Folding} of Class VII\indexvii\ orbits.  I find clockwise loops\index{Hysteresis} in 9 out of 11 Class VII orbits.  In the remaining two observations, the oscillations were extremely fast.  As a result, the errors in the HID$_1$\index{Hardness-intensity diagram} of these two observations were too large to succesfully select peaks, and I am unable to confirm or reject the presence of loops.

\begin{figure}
    \includegraphics[width=0.8\columnwidth, trim = 0.6cm 0 3.9cm 0]{images/N_sgram.png}
    \captionsetup{singlelinecheck=off}
    \caption[A sliding window Lomb-Scargle spectrogram of Class VII observation 96420-01-18-05.]{A sliding window Lomb-Scargle periodogram\index{Lomb-Scargle periodogram} of Class VII\indexvii\ observation 96420-01-18-05, showing power density spectra from an overlapping 32\,s window moved 1\,s at a time.  The peak frequency of this low frequency QPO\index{Quasi-periodic oscillation} itself appears to oscillate with a frequency of $\sim5$mHz.}
   \label{fig:spect}
\end{figure}

\subsubsection{Class VIII -- Figure \ref{fig:Omulti}}

\begin{figure}
    \includegraphics[width=0.8\columnwidth, trim = 0.6cm 0 3.9cm 0]{images/Omulti.png}
    \captionsetup{singlelinecheck=off}
    \caption[Characteristic lightcurves and a power spectrum of Type VIII variability.]{Plots of the Class VIII\indexviii\ observation 96420-01-19-03, orbit 0.  \textit{Top-left}: 300\,s lightcurve\index{Lightcurve} binned on 2 seconds to show lightcurve evolution.  \textit{Top-right}: Fourier Power Density Spectrum\index{Fourier analysis}.  \textit{Bottom}: Lightcurve binned on 0.5 seconds.  Inset is a zoom of the 25\,s portion of the lightcurve highlighted in cyan, to show the second-scale structure in the lightcurve.}
   \label{fig:Omulti}
\end{figure}

\par The lightcurve\index{Lightcurve} of this variability class\indexviii\ shows the dipping\index{Dip} behaviour seen in Class VI\indexvi, as can be seen in Figure \ref{fig:Omulti} at $t\approx125$--$150$\,s.  The dips are less frequent than in Class VI.  The behaviour outside of the dips is dominated by highly structured high-amplitude oscillations consisting of flares\index{Flare} with a peak to peak separation of $3.4\pm1.0$\,s.  The PDS\index{Fourier analysis} shows this behaviour as a very significant ($q$-value > 20\indexq) QPO\index{Quasi-periodic oscillation}; two harmonics\index{Harmonic} of this QPO are also visible.  The PDS also shows a strong ($q$-value$\sim5$\indexq) QPO at $\sim9$\,Hz.
\par I attempted to fold\index{Folding} Class VIII\indexviii\ lightcurves, ignoring the portions of data corresponding to dips\index{Dip}, using my flare-finding algorithm.  The high frequency of the dominant oscillation in Class VIII resulted in large errors in the peak times of individual flares, which translated to large errors in all HID$_1$s\index{Hardness-intensity diagram}; however, I was able to ascertain the presence in loops\index{Hysteresis} in 8 out of 16 orbits.  All 8 of these loops are clockwise.

\subsubsection{Class IX -- Figure \ref{fig:Qmulti}}
\label{sec:ClassIX}
\begin{figure}
    \includegraphics[width=0.8\columnwidth, trim = 0.6cm 0 3.9cm 0]{images/Qmulti.png}
    \captionsetup{singlelinecheck=off}
    \caption[Characteristic lightcurves and a power spectrum of Type IX variability.]{Plots of the Class IX\indexix\ observation 96420-01-35-02, orbit 1.  \textit{Top-left}: 1200\,s lightcurve\index{Lightcurve} binned on 2 seconds to show lightcurve evolution.  \textit{Top-right}: Fourier Power Density Spectrum\index{Fourier analysis}.  \textit{Bottom}: Lightcurve binned on 2 seconds.}
   \label{fig:Qmulti}
\end{figure}

\par The 1\,s lightcurve\index{Lightcurve} of a Class IX\indexix\ observation is superficially similar to the lightcurve of a Class I\indexi\ observation, with little obvious structured variability\index{Variability} at timescales larger than 2\,s; however, large count rate dips\index{Dip} like those seen in Classes VI\indexvi\ and VIII\indexviii\ (e.g. the feature at $t\approx410$\,s in the lightcurve of Figure \ref{fig:Qmulti}) are very occasionally observed.  These dips may in turn be coupled to short second-scale flares\index{Flare} in which count rate briefly increases by a factor of 2--3.
\par Outside of these dips and flares, the lightcurve of a Class IX\indexix\ observation is indistinguishable from the lightcurve\index{Lightcurve} of a Class I\indexi\ or Class II\indexii\ observation.  However, in Figure \ref{fig:IIIisHarder}, I show that Class IX occupies a very different part of the global $H_{A2}$/$H_{A1}$ colour-colour diagram\index{Colour-colour diagram}.  Class IX observations show a significantly larger $H_{A2}$\index{Colour} than Class I and II observations, but a significantly lower $H_{A1}$.
\par The PDS\index{Fourier analysis}\index{PDS|see {Fourier analysis}} reveals significant broad band noise peaked at $\sim$0.3 Hz, and the $\sim5$\,Hz QPO seen in other classes is absent.  \citet{Altamirano_HFQPO} discovered high frequency ($\sim66$\,Hz) QPOs\index{Quasi-periodic oscillation} in observations corresponding to this variability class.

\subsection{Swift}

\par Observations with \indexswift\textit{Swift} took place throughout the 2011-2013 outburst of IGR J17091-3624\index{IGR J17091-3624}.  Between MJDs 55622 and 55880, 17 \indexxrt\textit{Swift/XRT} were at least partly simultaneous with an \rxte\indexrxte\ observation, corresponding to at least one observation of all 9 classes.  In each case, the \textit{Swift} and \rxte\ lightcurves were similar.  The remainder of the \textit{Swift/XRT} observations during this time were also consistent with belonging to one of my nine classes.  Given that the \rxte\ data have higher count rate and time resolution, I do not further discuss the \textit{Swift} observations taken before MJD 55880.
\par Between MJD 55952 and 56445, \indexswift\textit{Swift} observations showed IGR J17091-3624\index{IGR J17091-3624} decreasing in flux.  For all observations longer than 500\,s, I rebinned the lightcurves to 10\,s and calculated the fractional RMS\index{RMS}.  I find the lower and upper quartiles of the fractional RMS in these measurements to be 18.3\% and 21.7\% respectively.  \textit{INTEGRAL} observations taken as part of a scan programme of the Galactic Plane \citep{Fiocchi_PlaneScan} and reported by \citet{Drave_Return} suggest that IGR J17091-3624 returned to the hard state\index{Low/Hard state} between MJDs 55952 and 55989.  Therefore these observations sample IGR J17091-3624 in the hard state.

\subsection{INTEGRAL}

\par The results of the \indexintegral\indexibis\textit{INTEGRAL}/IBIS analysis are presented in Table \ref{tab:IBIS_results}. \textsf{C.B.} finds clear detections of IGR J17091-3624\index{IGR J17091-3624} in all energy bands during the hardest period (MJD 55575--55625) of the 2011--2013 outburst\index{Outburst}. Conversion from detected counts to flux was achieved using an \textit{INTEGRAL}/IBIS observation of the Crab taken between MJD 57305.334 and 57305.894. Conversion from Crab units to standard flux units was obtained by conversion factors listed in \citet{Bird_Survey} and \citet{Bazzano_Survey}.

\begin{table*}
\begin{tabular}{cccccc}
\hline
\hline
Energy 		& Intensity 		& Significance 	& Exposure 	& Flux 				& Flux					\\
(keV)		& (cts/s)			& $\sigma$		& (ks)		& (mCrab) 			& (10$^{-10}$ergs~s$^{-1}$~cm$^{-2}$) 	\\
\hline
20--40		& 12.39$\pm$0.05	& 247			& 115		& 93.5$\pm$0.38		& 7.08$\pm$0.03			\\
40--100		& 7.06$\pm$0.05		& 157			& 163		& 83.5$\pm$0.60		& 7.87$\pm$0.06			\\
100--150	& 1.05$\pm$0.03		& 40			& 173		& 66.9$\pm$1.91		& 2.14$\pm$0.06			\\
150--300	& 0.23$\pm$0.03		& 7.6			& 179		& 46.6$\pm$5.96		& 2.24$\pm$0.29			\\	
\hline
\hline
\end{tabular}
\caption[Results from the IBIS/ISGRI analysis of the 2011--2013 Outburst of IGR J17091.]{Results from the \indexibis IBIS/ISGRI analysis of the 2011--2013 Outburst\index{Outburst} of IGR J17091\index{IGR J17091-3624}. The 20--40\,keV flux is given in units of mCrab and (10$^{-11}$\ergf ). Conversion between counts and mCrab was obtained using an observation of the Crab taken during Revolution 1597 between MJD 57305.334 and 57305.894 and the conversion factors of \citet{Bird_Survey} and \citet{Bazzano_Survey}.}
\label{tab:IBIS_results}
\end{table*}

\par Comparing these results with those of \citet{Bazzano_Survey}, we see that IGR J17091\index{IGR J17091-3624} is detected for the first time above 150\,keV with a detection significance of 7.6\,$\sigma$, corresponding to a flux of $2.24\pm0.29\times10^{-10}$\ergf\ (Figure \ref{fig:sigmap}).

\begin{figure}
    \includegraphics[width=0.7\columnwidth, trim = 0.6cm 0 3.9cm 0]{images/sigmap.png}
    \captionsetup{singlelinecheck=off}
    \caption[\textit{INTEGRAL}/ISGRI 150--300\,keV significance map of a $2^\circ$ region centred on the position of IGR J17091-3624.]{\indexibis\textit{INTEGRAL}/ISGRI 150--300\,keV significance map of a $2^\circ$ region centred on the position of IGR J17091-3624\index{IGR J17091-3624}, showing the first significant detection of this source above 150\,keV.  The detection significance is 7.6 $\sigma$.}
   \label{fig:sigmap}
\end{figure}

\subsection{Chandra}

\par In Figure \ref{fig:Cha_lc}, I present lightcurves from the three \indexchandra\textit{Chandra} observations considered in this chapter (see also Table \ref{tab:Chandra} for details of these observations).

\begin{figure}
    \includegraphics[width=0.8\columnwidth, trim = 0.6cm 0 3.9cm 0]{images/Chandra.eps}
    \captionsetup{singlelinecheck=off}
    \caption[\textit{Chandra} lightcurves showing examples of Class I, VII and IX variability.]{1 ks segments of lightcurves taken from \indexchandra\textit{Chandra} observations 12505, 12405 and 12406, showing Class I\indexi, Class VII\indexvii\ and Class IX\indexix\ variability respectively.  The lightcurve presented for observation 12505 is for the energy range 0.06-10\,keV, while the other two lightcurves are for the energy range 0.5-10\,keV.  All three lightcurves are binned to 0.5\,s.}
   \label{fig:Cha_lc}
\end{figure}

\par Observation 12505 was performed within 24 hours of \indexrxte\rxte\ observation 96420-01-02-01, which showed Class I\indexi\ variability.  No structured variability is seen in the lightcurve of ObsID 12505 (Figure \ref{fig:Cha_lc}, upper panel), which is consistent with Class I.  Note that I consider the energy range 0.06-10\,keV for this observation but 0.5-10\,keV for observations 12405 and 12406.
\par Observation 12405 was performed within 24 hours of \indexrxte\rxte\ observation 96420-01-23-03, which showed Class V\indexv\ variability.  The two observations were not simultaneous; ObsID 12405 began $\sim8.4$ ks after ObsID 96420-01-2303 finished.  The lightcurve of \indexchandra\textit{Chandra} ObsID 12405 (shown in Figure \ref{fig:Cha_lc}, middle panel) shows a mean count rate of 41\,cts\,s$^{-1}$.  The lightcurve shows fast flaring\index{Flare} behaviour (with a recurrence time\index{Recurrence time} on the order of 10s of seconds) in which the frequency changes widely on timescales of $\sim1000$\,s.  This observation strongly resembles a Class VII\indexvii\ lightcurve, but with its characteristic timescales increased by a factor of $\sim4$.  This leads to the possibility that the low number of Class VII\indexvii\ \rxte\ observations I identify is due to a selection effect; we would not have been able to see this observation's long-term Class VII-like behaviour if the observation had been shorter than $\sim2$ ks.
\par Observation 12406 was performed within 24 hours of \indexrxte\rxte\ observation 96420-01-32-06, which showed Class IX\indexix\ variability.  The lightcurve presented for \indexchandra\textit{Chandra} ObsID 12406 shows a mean count rate (36 cts s$^{-1}$), which is consistent with IGR J17091\index{IGR J17091-3624} being harder in this observation than in Observation 12505.  This, combined with the lack of variability seen in its lightcurve, suggests that Observation 12505 is consistent with Class IX.

\subsection{XMM-Newton}

\par In Figure \ref{fig:XMM} I show lightcurves from two \indexxmm\textit{XMM-Newton} observations.  The lightcurve of \textit{XMM-Newton} observation 0677980201, shown in the upper panel of Figure \ref{fig:XMM}, shows the regular flares characteristic of Class IV variability.  A simultaneous \indexrxte\rxte\ observation (ObsID 96420-01-05-000) also showed Class IV\indexiv\ variability.

\begin{figure}
    \includegraphics[width=0.8\columnwidth, trim = 0.6cm 0 3.9cm 0]{images/xmmlc.eps}
    \captionsetup{singlelinecheck=off}
    \caption[\textit{XMM-Newton} lightcurves showing an example of Class IV variability and the hard state.]{Lightcurves of \indexxmm\textit{XMM-Newton} observations 0677980201 and 0700381301, showing Class IV\indexiv\ variability and the hard\index{Low/Hard state} state respectively.  Both lightcurves binned to 2\,s.  Data for observation 0677980201 is taken from \indexepic\textit{EPIC-MOS2} and data for observation 0700381301 is taken from \textit{EPIC-pn}.}
   \label{fig:XMM}
\end{figure}

\par \indexxmm\textit{XMM-Newton} observation 070038130, shown in the lower panel of Figure \ref{fig:XMM}, was made after the end of \indexrxte\rxte\ observations IGR J17091-3624\index{IGR J17091-3624}.  As such it cannot be compared with contemporaneous \rxte\ data.  The 5\,s binned lightcurve shows no apparent variability, but a Fourier PDS of the observation (shown in Figure \ref{fig:xmmqpo}) reveals a QPO\index{Quasi-periodic oscillation} centred at around $\sim0.15$\,Hz and a broad band noise component at lower frequencies.  \citet{Drave_Return} reported that IGR J17091 transited to the hard state\index{Low/Hard state} in February 2012, seven months before this observation was taken.  As such, I find that observation 0677980201 samples the hard state in IGR J17091 and is thus beyond the scope of my set of variability classes.

\begin{figure}
    \includegraphics[width=0.9\columnwidth, trim = 0cm 0cm 0.5cm 1.0cm, clip]{images/multipower.eps}
    \captionsetup{singlelinecheck=off}
    \caption[$\nu P(\nu)$-normalised co-added power density spectra of \textit{XMM-Newton} observation 0700381301 and \textit{Suzaku} observation 407037010.]{RMS-normalised\index{RMS normalisation} co-added power density spectra\index{Fourier analysis} of \indexxmm\textit{XMM-Newton} observation 0700381301 and \indexsuzaku\textit{Suzaku} observation 407037010.  Both observations were taken simultaneously on September 29 2012 (MJD 56199).  I sample observation 0700381301 up to a frequency of 10\,Hz, while the 2\,s time resolution of observation 407037010 results in a Nyquist frequency\index{Nyquist frequency} of 0.25\,Hz.}
   \label{fig:xmmqpo}
\end{figure}

\subsection{\textit{Suzaku}}

\par The two \indexsuzaku\textit{Suzaku} observations of IGR J17091-3624\index{IGR J17091-3624} considered here, ObsIDs 407037010 and 407037020, were performed during the 2nd and 3rd re-flares\index{Re-flare} of the hard state\index{Low/Hard state} phase of the 2011--2013 outburst.  ObsID 407037010 was taken simultaneously with \indexxmm\textit{XMM-Newton} observation 0700381301.  The\indexxis\ XIS 0 count rates are 7.8 cts\,s$^{-1}$ and 2.5\,cts\,s$^{-1}$ respectively.
\par Neither lightcurve shows `heartbeats'\indexrho\ or any other type of GRS 1915-like\index{GRS 1915+105} variability\index{Variability}.  However, \textsf{K.Y.} and I find evidence of a low frequency QPO\index{Quasi-periodic oscillation} feature at $\sim$0.15 Hz in the ObsID 407037010; this QPO is also seen in \indexxmm\textit{XMM-Newton} observation 0700381301 (Figure \ref{fig:xmmqpo}).  The presence of a QPO below 1\,Hz and flat-topped power density spectrum\index{Fourier analysis} confirm that IGR J17091\index{IGR J17091-3624} was in the hard state\index{Low/Hard state} at this time.

\section{Discussion}

\par Using observations from \indexxmm\textit{XMM-Newton}, \indexrxte\rxte\ and \indexchandra\textit{Chandra}, I describe the complex variability\index{Variability} seen in IGR J17091\index{IGR J17091-3624} as a set of nine variability `classes'\index{Variability class}, labelled I to IX.  These classes are distinguished from each other by values of upper and lower quartile (i.e. 25\textsuperscript{th} and 75\textsuperscript{th} percentile) count rates, mean RMS\index{RMS}, the presence of QPOs\index{Quasi-periodic oscillation} in Fourier PDS\index{Fourier analysis}, the shape of flare\index{Flare} and dip\index{Dip} features in the lightcurve\index{Lightcurve} and the presence of loops\index{Hysteresis} in the 6--16/2--6 keV hardness-intensity diagram HID$_1$\index{Hardness-intensity diagram}.  See Section \ref{sec:results} for a full description of these classes.
\par The classification of some observations is clearer than others.  Some orbits were too short to definitively quantify the behaviour of the source, whereas some other orbits contain a transition between two classes\index{Variability class}.  An example lightcurve\index{Lightcurve} showing a transition from Class III\indexiii\ to Class IV\indexiv\ is presented in Figure \ref{fig:HybridClasses}.

\begin{figure}
    \includegraphics[width=\columnwidth, trim =0cm 0 0cm 0]{images/mixJandK.png}
    \captionsetup{singlelinecheck=off}
    \caption[A lightcurve of observation 96420-01-06-02, showing a transition in behaviour between Classes IV and V.]{A lightcurve\index{Lightcurve} of observation 96420-01-06-02, orbit 0, showing a transition in behaviour between Class III\indexiii\ (in cyan, see Section \ref{sec:classIII}) and Class IV\indexiv\ (in red, see Section \ref{sec:classIV}).}
   \label{fig:HybridClasses}
\end{figure}

\par My set of classes\index{Variability class} is analogous to, but not based upon, the set of variability classes defined by \citealt{Belloni_GRS_MI} to describe the behaviour of the similarly complex LMXB\index{X-ray binary!Low mass} GRS 1915\index{GRS 1915+105}.  This ensures that my set of classes is not biased by an \textit{a priori} assumption that the two objects are similar.  However if we do assume that wide range of variability\index{Variability} seen in these two objects are driven by the same physical processes, a direct comparison between the variability classes in the two systems can further our understanding of the physics that drive these exotic objects.
%\par I also use all 2011-2013 IGR J17091-3624 data from \rxte , \textit{XMM-Newton}, \textit{Chandra}, \textit{Swift}, \textit{INTEGRAL} and \textit{Suzaku} to analyse the long-term evolution of the 2011--2013 outburst.  This in turn corresponds to all available X-ray data taken during this outburst.

\subsection{Variability Classes: IGR J17091 vs. GRS 1915}

\label{sec:IGRcomp}

\par As observations of IGR J17091\index{IGR J17091-3624} and GRS 1915\index{GRS 1915+105} suffer from different values of interstellar absorption\index{NH@$N_H$}\index{Absorption|see {$N_H$}} $N_H$\footnote{$N_H$, or the interstellar absorption, is a measure of the surface density of hydrogen atoms along a column between the object in question and the Earth.  A high value of $N_H$ causes low-energy X-rays to be supressed more than high-energy X-rays, increasing the apparent colour of a source.  $N_H$ can be estimate by fitting models to the energy spectrum of a source.}, I cannot directly compare the absolute colours\index{Colour} of these two objects.  However, I can compare the evolution of colour both over time and as a function of count rate.  I therefore use these parameters, along with power spectra\index{Fourier analysis} and lightcurve\index{Lightcurve} morphology, when comparing GRS 1915 with IGR J17091.
\par For seven of my classes\index{Variability class}, I was able to assign the closest matching class described by \citealt{Belloni_GRS_MI} for GRS 1915\index{GRS 1915+105} (see Table \ref{tab:class_assign}).  I am unable to find analogues to my classes VII\indexvii\ and VIII\indexviii\ in observations of GRS 1915, and I suggest that these classes are unique to IGR J17091\index{IGR J17091-3624}.

\begin{table}
\centering
\caption[The nine variability classes of IGR J17091-3624, showing the name of the closest corresponding variability class in GRS 1915+105.]{The nine variability classes\index{Variability class} of IGR J17091-3624\index{IGR J17091-3624}, showing the name of the closest corresponding variability class in GRS 1915+105\index{GRS 1915+105}.  The names of GRS 1915+105 classes are taken from \citet{Belloni_GRS_MI}, where more detailed descriptions can be found.  Eight additional classes of GRS 1915+105 have been described; I do not find analogies to these classes in IGR J17091-3624.}
\label{tab:class_assign}
\begin{tabular}{cc} % four columns, alignment for each
\hline
\hline
IGR J17091-3624 Class & GRS 1915+105 Class\\
\hline
I\indexi&$\chi\indexchi$\\
II\indexii&$\phi\indexphi$\\
III\indexiii&$\nu\indexnu$\\
IV\indexiv&$\rho\indexrho$\\
V\indexv&$\mu$\indexmu\\
VI\indexvi&$\lambda\indexlambda$\\
VII\indexvii&\textit{None}\\
VIII\indexviii&\textit{None}\\
IX\indexix&$\gamma\indexgamma$\\
\hline
\hline
\end{tabular}
\end{table}

\par Below, I evaluate my mapping between GRS 1915\index{GRS 1915+105} and IGR J17091\index{IGR J17091-3624} classes\index{Variability class}, and interpret the differences between each matched pair.

\subsubsection{Classes I and II -- Figures \ref{fig:Bmulti}, \ref{fig:Emulti}}

\label{sec:DisI}

\par Classes I\indexi\ and II\indexii\ both show low count rates and little structure in their lightcurves\index{Lightcurve}.  The two classes in GRS 1915\index{GRS 1915+105} that also show this lightcurve behaviour are Class $\chi$\indexchi\footnote{Note that, in GRS 1915+105, Class $\chi$ is further subdivided into four classes based on hard colour \citep{Belloni_GRS_MI,Pahari_Chi}.  As I cannot obtain hard colour\index{Colour} for IGR J17091\index{IGR J17091-3624}, I treat $\chi$ as a single variability class here.} and Class $\phi\indexphi$.  \citealt{Belloni_GRS_MI} differentiate between Classes $\phi$ and $\chi$ based on the hard colour (corresponding to $C_2$), as Class $\chi$ has a significantly higher value for this colour than Class $\phi$.

\par Data from \indexrxte\rxte\  indicates that the transition from the hard\index{Low/Hard state} state to the soft intermediate\index{High/Soft state} state between MJDs 55612 and 55615 \citep{Drave_Return}.  This was confirmed by a radio spectrum taken on MJD 55623 which was consistent with an observation of discrete ejecta \citep{Rodriguez_D}.  This observation of discrete ejecta at the transition between the hard state and the intermediate state has been reported in other LMXBS\index{X-ray binary!Low mass} (e.g. XTE J1550-564\index{XTE J1550-564}, \citealp{Rodriguez_XTE}), and has also been associated with transitions to the $\chi$\indexchi\ Class in GRS 1915\index{GRS 1915+105} (\citealp{Rodriguez_Ejection}, see also review by \citealp{Fender_Jets}).

\par Using Fourier PDS\index{Fourier analysis}, I conclude that Class I\indexi\ is analogous to Class $\chi$\indexchi\ in GRS 1915, while Class II\indexii\ is analogous to Class $\phi$\indexphi.  In Class $\chi$ observations of GRS 1915, broad band noise between $\sim1-10$\,Hz and a QPO\index{Quasi-periodic oscillation} at around 5\,Hz are seen in the PDS.  I find that both of these are present in Class I observations of IGR J17091\index{IGR J17091-3624}.  On the other hand, I find that Class $\phi$ observations of GRS 1915\index{GRS 1915+105} do not show this broad band noise, and show either a weak ($q$-value\indexq\ $\lesssim 3$) QPO at $\sim5$\,Hz or no QPO at all.  I find that the weak QPO and lack of broad band noise are also seen in the PDS of Class II observations.

\subsubsection{Classes III and IV -- Figures \ref{fig:Gmulti}, \ref{fig:Jmulti}}

\par Classes III\indexiii\ and IV\indexiv\ both show highly regular flaring\index{Flare} activity in their lightcurves\index{Lightcurve}, but they differ in terms of timescale and pulse profile.  As can be seen in lightcurves in Figure \ref{fig:Jmulti}, flares in Class IV occur every $\sim32$\,s and are nearly identical to each other in shape.  On the other hand, as can be seen in Figure \ref{fig:Gmulti}, flares in Class III occur every $\sim61$\,s and may or may not end in a much faster sharp peak which is never seen in Class IV.  In Figure \ref{fig:III_IV_burst} I show a two-dimensional histogram of flare peak count rate against flare duration, showing all flares in all observations classified as Class III or Class IV.  In this figure, I can see that flares tend to group in one of two regions in count rate-duration space; a region between $\sim90\mbox{--}110$ \spcu and $\sim35\mbox{--}55$\,s, corresponding to flares seen in Class III, and a region between $\sim150\mbox{--}250$ \spcu and $\sim20\mbox{--}55$\,s, corresponding to flares seen in Class IV.  From this plot, I conclude that the flares seen in Class III exist in a different population\index{Population study} to the flares seen in Class IV.
\par The GRS 1915\index{GRS 1915+105} classes that show behaviour most similar to these are $\rho$\indexrho\ and $\nu$\indexnu; both produce similar structures in their lightcurve, but Class $\nu$ is differentiated from Class $\rho$ by the presence of a secondary count rate peak which occurs $\sim5$\,s after the primary \citep{Belloni_GRS_MI}.

\begin{figure}
    \includegraphics[width=\columnwidth, trim = 0mm 0mm 0mm 0mm]{images/GJBurst.png}
    \captionsetup{singlelinecheck=off}
    \caption[Every flare in all observations identified as Class III or Class IV, plotted in a two-dimensional histogram of flare peak count rate against flare duration to show the two-population nature of these events.]{Every flare\index{Flare} in all observations identified as Class III\indexiii\ or Class IV\indexiv, plotted in a two-dimensional histogram of flare peak count rate against flare duration to show the two-population\index{Population study} nature of these events.  Flares belonging to Class IV occupy the distribution at higher peak rate and lower duration, whereas flares belonging to Class III occupy the distribution at lower peak rate and higher duration.}
   \label{fig:III_IV_burst}
\end{figure}

\par The secondary peak is present in most Class III observations and some Class IV observations (Figure \ref{fig:III_IV_spike}), suggesting that both classes consist of a mix of $\rho$-like and $\nu$-like observations.  However, the poor statistics sometimes make the presence of this secondary peak difficult to detect.  As such, I do not use the presence or absence of this peak as a criterion when assigning classes.  Instead I choose to separate Classes III and IV based on the larger-scale structure in their lightcurves (see Section \ref{sec:classIV}).  Due to the aforementioned difference in burst populations\index{Population study} between the two classes, I suggest that classes III and IV do represent two distinct classes rather than a single class with a period that drifts over time.  I suggest that Classes $\rho$ and $\nu$ in GRS 1915 could also be re-partitioned in this way.

\begin{figure}
    \includegraphics[width=\columnwidth, trim = 0mm 0mm 0mm 0mm]{images/classIIIsecpeak.png}
    \captionsetup{singlelinecheck=off}
    \caption[Lightcurve from Class III observation 96420-01-10-01 of IGR J17091-3624, with pairs of primary and secondary count rate spikes highlighted.]{Lightcurve\index{Lightcurve} from Class III\indexiii\ observation 96420-01-10-01 of IGR J17091-3624\index{IGR J17091-3624}, with pairs of primary and secondary count rate spikes highlighted in cyan and red respectively.  The yellow region highlights a primary count rate spike that did not produce a secondary.}
   \label{fig:III_IV_spike}
\end{figure}

\par However, HID$_1$\index{Hardness-intensity diagram} loops\index{Hysteresis} are found to generally execute in an anticlockwise direction in Classes III\indexiii\ and IV\indexiv\ (previously noted by e.g. \citealp{Altamirano_IGR_FH}); the opposite direction to the clockwise loops in Classes $\rho$\indexrho\ and $\nu$\indexnu\ reported by e.g. \citealp{Belloni_GRS_MI} and repeated by us using the same method I apply to data from IGR J17091-3624\index{IGR J17091-3624} (see Section \ref{sec:dex}).  This suggests that Classes III and IV could be generated by a different physical mechanism to Classes $\rho$ and $\nu$.  Alternatively, Classes III and IV could be generated by the same mechanism as $\rho$ and $\nu$ if some other unknown process was able to alter the spectral evolution of flares\index{Flare} in these classes.

\subsubsection{Class V -- Figure \ref{fig:Kmulti}}

\par The lightcurve\index{Lightcurve} of a Class V\indexv\ observation appears similar to that of a Class $\mu$\indexmu\ observation of GRS 1915\index{GRS 1915+105}, as both are characterised by rapid $\rho$\indexrho-like flares which occur less regularly than in Class $\rho$.  In addition to this, flares in Class $\mu$ fall into two clear populations\index{Population study}, as do the flares in Class V.  However, significant differences exist between Class V and Class $\mu$.  Class $\mu$ observations are characterised by long ($\sim100$ s) excursions to plateaus\index{Plateau} of high count rate, a behaviour which is not seen in any Class V observation thus far.
\par I note that the HID$_1$\index{Hardness-intensity diagram} in Class V\indexv\ observations displays a loop\index{Hysteresis} in the clockwise direction; the opposite direction to the looping seen in Classes III\indexiii\ and IV\indexiv\ but the same direction seen in Class $\mu$\indexmu.
\par Regarding the two-population\index{Population study} nature of flares\index{Flare} seen in this class (see Section \ref{sec:classV}), I suggest that V$_2$ flares may simply be two V$_1$ flares that occur close together in time, such that the second flare starts during the decay of the first flare.  This would result in an apparent two-peaked flare structure, as we see in type V$_2$ flares.  This interpretation also accounts for the bimodal distribution of flare duarations shown in the 2D histogram of Figure \ref{fig:two_popV}, as this could be caused by the misinterpretation of two-flare V$_2$ events as a single event.  This also accounts for the Gaussian distribution of peak flare intensities seen in Figure \ref{fig:two_popV}), as the constituents of each V$_2$ event would be from the same population as V$_1$ flares.

\subsubsection{Class VI -- Figure \ref{fig:Lmulti}}

\par Class VI\indexvi\ is dominated by long flaring\index{Flare} periods which separate periods of low count rate, as can be seen in the lightcurve\index{Lightcurve} presented in Figure \ref{fig:Lmulti}.  Similar behaviour is seen in the lightcurves of observations of GRS 1915 belonging to Classes $\lambda$\indexlambda\ and $\omega$\indexomega\ \citep{KleinWolt_OmegaClass}.  However, the long count rate `dips'\index{Dip} are far less regular in Class VI than in Classes $\lambda$ and $\omega$, and I also note long periods of medium count rate during which neither flares nor dips occur.  This variability class\index{Variability class} is noted by \citet{Pahari_IGRClasses} who suggest that this class is unique to IGR J17091\index{IGR J17091-3624}\footnote{\citet{Pahari_IGRClasses} refers to Class VI as Class C2.}.  However, \citet{Pahari_ClassVI} show that, in a plot of burst decay time against burst rise time, Classes VI and $\lambda$ fall in a straight line, suggesting a similar physical origin for both.
\par While it is cetainly true that Class VI\indexvi\ is not a perfect analogue of either Class $\lambda$\indexlambda\ or Class $\omega$\indexomega\, Class VI only differs noticeably from Class $\lambda$ during the extended low-variability portions of its lightcurves.  As such, I associate Class VI with Class $\lambda$.

\subsubsection{Class VII -- Figure \ref{fig:Nmulti}}

\par I am unable to find an analogue of Class VII\indexvii\ in observations of GRS 1915\index{GRS 1915+105}.  This class, and its apparent uniqueness, have previously been noted by \citealp{Pahari_IGRClasses}\footnote{\citet{Pahari_IGRClasses} refers to Class VII as Class C1.}.  \citeauthor{Pahari_IGRClasses} found  that the $C_2$ hard colour\index{Colour} in this class increases during count rate dips and decreases during count rate peaks.  Here I reproduced the results of \citeauthor{Pahari_IGRClasses} and found that the anti-correlation between hard-colour and intensity is not physical, but due to the definition of $C_2$: the count rate in band $L_C$ is approximately constant and consistent with background, and therefore $C_2=L_C/L_A \propto L_A^{-1}$, which will naturally anticorrelate with intensity.
%\par Although a correlation between QPO\index{Quasi-periodic oscillation} frequency and count rate has been noted in the $\sim5$\,Hz QPO seen in GRS 1915 (e.g. \citealp{Markwardt_FluxFreqGRS,Vignarca_FluxFreqGRS}), this QPO is also seen in Class VII observations at the same time as the $\sim0.1$\,Hz QPO.  As such, the flux-frequency relationship in the very low frequency ($\sim0.1$\,Hz) QPO in Class VII is apparently unique amongst the classes of both IGR J17091 and GRS 1915.

\subsubsection{Class VIII -- Figure \ref{fig:Omulti}}

\par I am unable to find an analogue of Class VIII\indexviii\ in observations of GRS 1915\index{GRS 1915+105}.  When it is flaring\index{Flare}, the lightcurve\index{Lightcurve} waveform is similar to that seen in Class $\rho$\indexrho, with rapid regular spikes in count rate.  The lightcurve also shows irregular dips\index{Dip} in count rate similar to those seen in Class VI\indexvi\ and in Class $\lambda$\indexlambda\ in GRS 1915.
\par However, the amplitude of the flares in Class VIII\indexviii\ is much larger, and the frequency much higher, than in Classes VI\indexvi\ or $\lambda$\indexlambda.  The amplitude of the flares\index{Flare} in Class VIII can approach $\sim350$\,cts s$^{-1}$\,PCU$^{-1}$, while the flare separation time of 4--5\,s makes Class VIII the fastest flaring activity seen in any class of IGR J17091\index{IGR J17091-3624} or GRS 1915\index{GRS 1915+105}.  As such, I consider this variability class distinct from both Class VI and Class $\lambda$. 

\subsubsection{Class IX - Figure \ref{fig:Qmulti}}

\label{sec:DisIX}

\par Class IX\indexix\ is defined by long periods of high amplitude but unstructured variability\index{Variability} (with a broad peaked noise component in the Fourier spectrum\index{Fourier analysis} peaked at $\sim$0.3 Hz) punctuated with infrequent irregular short-duration `spikes' in which the count rate increases by a factor of $\sim2$--$3$.  A similarity between this Class and Class $\gamma$\indexgamma\ in GRS 1915\index{GRS 1915+105} has been previously noted by \citet{Altamirano_HFQPO}.  However, the irregular spikes seen in some Class IX lightcurves are not reproduced in Class $\gamma$ lightcurves of GRS 1915.

\subsection{General Comparison with GRS 1915+105}

\par Overall, variability\index{Variability} in IGR J17091\index{IGR J17091-3624} tends to be faster than structurally similar variability in GRS 1915\index{GRS 1915+105}, as can be noted in Classes III\indexiii\ and IV\indexiv\ compared to Classes $\rho$\indexrho\ and $\nu$\indexnu\ (see also \citealp{Altamirano_IGR_FH}).  Additionally, IGR J17091 also displays highly structured variability unlike anything yet seen in GRS 1915, with classes VII\indexvii\ and VIII\indexviii\ in particular showing very fine detail in their lightcurves.
\par In total I find 2 variability classes\index{Variability class} which are seen in IGR J17091\index{IGR J17091-3624} but not in GRS 1915\index{GRS 1915+105}, compared with 8 that are seen in GRS 1915 but not in IGR J17091.  As relatively little data exists on GRS 1915-like variability in IGR J17091, the presence of classes in GRS 1915 that are not seen in IGR J17091 could simply be an observational effect.  It is unknown how long each variability class lasts for and, as such, additional variability classes could have occurred entirely while IGR J17091 was not being observed.  However, GRS 1915 has displayed variability classes consistently since its discovery in 1992 (see e.g. see \citealp{Huppenkothen_ML}), implying that the two classes seen only in IGR J17091 are either completely absent in GRS 1915 or that they occur with a much lower probability.  In either case, this implies physical differences between methods of generating GRS 1915-like variability\index{Variability} in the two objects.  
\par As noted in section \ref{sec:DisI}, variability classes\index{Variability class} seen in both IGR J17091\index{IGR J17091-3624} and GRS 1915\index{GRS 1915+105} show differences between the different objects.  In particular, I note the presence of irregular flares\index{Flare} in Class IX\indexix\ which are not seen in the analogous Class $\gamma$\indexgamma.  If these classes are indeed generated by the same processes in both objects, the differences between them must represent physical differences between the objects themselves.
\par It has previously been noted that, while the hardness ratios\index{Colour} in IGR J17091\index{IGR J17091-3624} and GRS 1915\index{GRS 1915+105} during $\rho$\indexrho-like classes are different, the fractional hardening between the dip\index{Dip} and peak of each flare\index{Flare} is consistent with being the same in both objects \citep{Capitanio_peculiar}.  This suggests that the same physical process is behind the `heartbeats' seen in both objects.
\par I note the presence of hysteretic HID$_1$\index{Hardness-intensity diagram} loops\index{Hysteresis} in some classes\index{Variability class} of both objects.  Although these loops are always clockwise in GRS 1915\index{GRS 1915+105}, they can be executed in either direction in IGR J17091\index{IGR J17091-3624}.  Classes in IGR J17091 that show loops all have a preferred loop direction: anticlockwise in Classes III\indexiii\ and IV\indexiv\ and clockwise in classes V\indexv, VI\indexvi, VII\indexvii\ and VIII\indexviii.  In cases where the loop direction was opposite to that expected for a given class, loop detections were generally only marginally significant.  In particular, I note that Classes IV and V tend to show loops in opposite directions, despite the similarities between their lightcurves and the $\rho$, $\nu$ and $\mu$ classes in GRS 1915.   The fact that IGR J17091 can show HID$_1$ loops in both directions suggests that an increase in soft emission can either precede or lag\index{Hard lag} a correlated increase in hard emission from IGR J17091.  Whether soft emission precedes or lags hard emission is in turn is dependent on the variability class.
\par There are also non-trivial similarities between variability\index{Variability} in the two objects.  I note the presence of a $\sim5$\,Hz QPO\index{Quasi-periodic oscillation} in many of the classes seen in IGR J17091\index{IGR J17091-3624}, and this same 5\,Hz QPO is seen in data from GRS 1915\index{GRS 1915+105}.  Similarly \citet{Altamirano_HFQPO} reported the discovery of a 66\,Hz QPO in IGR J17091; a very similar frequency to the 67\,Hz QPO observed in GRS 1915 \citep{Morgan_QPO}.  It is not clear why these QPOs would exist at roughly the same frequencies in both objects when other variability in IGR J17091 tends to be faster.

\subsection{Comparison with the Rapid Burster}

\par In 2015, \citet{Bagnoli_RB} reported the discovery of two GRS 1915-like\index{GRS 1915+105} variability classes\index{Variability class} in the neutron star\index{Neutron star} binary\index{X-ray binary!Low mass} MXB 1730-335, also known as the `Rapid Burster'\index{Rapid Burster}.  Specifically, \citet{Bagnoli_RB} note the presence of variability similar to Classes $\rho$\indexrho\ and $\theta$\indextheta\ in GRS 1915.
\par Class $\theta$\indextheta-like variability\index{Variability}, seen in \indexrxte\rxte\ observation 92026-01-20-02 of the Rapid Burster\index{Rapid Burster}, is not closely matched by any of the classes I identify for IGR J17091\index{IGR J17091-3624}.  However, the lightcurves of a Class $\theta$ observation feature large dips\index{Dip} in count rate similar to those seen in Classes VI\indexvi\ and VIII\indexviii\ in IGR J17091.
\par Conversely, Class $\rho$-like\indexrho\ variability\index{Variability} is seen in all three objects.  \citet{Bagnoli_RB} note that the variability of the $\rho$-like flaring\index{Flare} is slower in the Rapid Burster\index{Rapid Burster} than in either GRS 1915\index{GRS 1915+105} or IGR J17091\index{IGR J17091-3624}. It has previously been suggested that the maximum rate of flaring in LMXBs\index{X-ray binary!Low mass} should be inversely proportional to the mass of the compact object\index{Compact object} (e.g. \citealp{Belloni_Timescales,Frank_Timescales}).  In this case, the fact that variability is faster in IGR J17091 than in GRS 1915 could simply be due to a lower black hole\index{Black hole} mass in the former object \citep{Altamirano_IGR_FH}.  However if variability in the Rapid Burster is assumed to be physically analogous to variability in these two black hole objects, then a correlation between central object mass and variability timescale no longer holds.

\subsection{Comparison with \citealp{Altamirano_IGR_FH}}

\label{sec:Alta}
\par \citet{Altamirano_IGR_FH} identify 5 GRS 1915\index{GRS 1915+105} variability classes\index{Variability class} in a subset of observations from the 2011-2013 outburst\index{Outburst} of IGR J17091\index{IGR J17091-3624}: six of these observations are presented in Table \ref{tab:me_Diego} along with the best-fit GRS 1915 class that I assign it in this chapter (see also Table \ref{tab:class_assign}).

\begin{table}
\centering
\caption[The six IGR J17091-3624 ObsIDs explicitly classified in \citet{Altamirano_IGR_FH}.]{The six IGR J17091-3624\index{IGR J17091-3624} ObsIDs explicitly classified in \citet{Altamirano_IGR_FH}.  I also present the GRS 1915\index{GRS 1915+105} class\index{Variability class} with which I implicitly label each ObsID in this chapter.}
\label{tab:me_Diego}
\begin{tabular}{ccc} % four columns, alignment for each
\hline
\hline
ObsID & Altamirano \textit{et al.}& My Class\\
&Class&(implied)\\
\hline
96420-01-04-03&$\alpha\indexalpha$&$\rho\indexrho/\nu\indexnu$\\
96420-01-05-00&$\nu$&$\rho/\nu$\\
96420-01-06-00&$\rho$&$\rho/\nu$\\
96420-01-07-01&$\rho$&$\mu\indexmu$\\
96420-01-08-03&$\beta\indexbeta/\lambda\indexlambda$&$\lambda$\\
96420-01-09-06&$\mu$&$\lambda$\\
\hline
\hline

\end{tabular}
\end{table}

\par I acknowledge differences between the classifications assigned by me and by \citet{Altamirano_IGR_FH}.  I ascribe these differences to the different approaches we have used to construct our classes\index{Variability class}.  In particular while I have constructed an independent set of variability classes for IGR J17091\index{IGR J17091-3624} which I have then compared to the \citeauthor{Belloni_GRS_MI} classes for GRS 1915\index{GRS 1915+105}, \citeauthor{Altamirano_IGR_FH} applied the \citeauthor{Belloni_GRS_MI} classes for GRS 1915 directly to IGR J17091.
\par In general, the variability classes\index{Variability class} I find to be present in IGR J17091\index{IGR J17091-3624} are broadly the same as those noted by \citet{Altamirano_IGR_FH}.  I do not associate any class with Class $\alpha$\indexalpha in GRS 1915\index{GRS 1915+105}, but I find examples of all of the other variability classes posited by \citeauthor{Altamirano_IGR_FH} to exist in IGR J17091.
\par \citealp{Altamirano_IGR_FH} noted the presence of an anticlockwise loop\index{Hysteresis} in the HID\index{Hardness-intensity diagram} of `heartbeat'\indexrho-like observations of IGR J17091\index{IGR J17091-3624}, opposed to the clockwise loop seen in HIDs of $\rho$-class observations of GRS 1915\index{GRS 1915+105}.  This is consistent with my finding that hysteretic loops in classes III\indexiii\ and IV\indexiv\ also tend to execute in an anticlockwise direction.  However, I additionally find that hysteretic loops in classes V\indexv, VI\indexvi, VII\indexvii\ and VIII\indexviii\ tend to execute in a clockwise direction.  This is also different from GRS 1915, in which the loop is executed in the same direction in all classes.  I also additionally report that clockwise loops tend to be more complex than anticlockwise loops in IGR J17091, with many showing a multi-lobed structure not seen in GRS 1915.  This apparent inconsistency between the objects strengthens the suggestion in \citealp{Altamirano_IGR_FH} that the heartbeat-like classes in GRS 1915 and IGR J17091 may be generated by physically different mechanisms.

\subsection{New Constraints on Accretion Rate, Mass \& Distance}
\label{sec:newmass}

\par The constraints that \citealp{Altamirano_IGR_FH} placed on the mass and distance of IGR J17091\index{IGR J17091-3624} assumed that the object emitted at its Eddington luminosity\index{Eddington limit} at the peak of the 2011--2013 outburst\index{Outburst}.  They report a peak 2--50\,keV flux of $4\times10^{-9}$\ergf\ during flares\index{Flare} in `heartbeat'\indexrho-like lightcurves\index{Lightcurve} during this time.  The correction factor\index{Bolometric correction factor} $C_{Bol,Peak}$ to convert 2--50\,keV flux to bolometric flux is not well constrained, but \citealp{Altamirano_IGR_FH} suggest an order-of-magnitude estimate of $\lesssim3$, corresponding to a peak bolometric flux of $\lesssim1.2\times10^{-8}$\ergf .
\par \citealp{Maccarone_2pct} performed a study of the soft\index{High/Soft state} to hard\index{Low/Hard state} transitions in 10 LMXBs\index{X-ray binary!Low mass} with well-constrained distances and compact object masses.  They found that all but one perform this transition at a luminosity consistent with between 1\% and 4\% of their Eddington limit\index{Eddington limit}.  By assuming that all LMXBs complete their soft-to-hard transitions at Eddington fractions of $\sim1--4$\%, it is then possible to estimate the Eddington fraction of an object at any point during its outburst, even if its distance and compact object mass are not known.
\par I use \indexswift\textit{Swift} observation 00031921058 taken on MJD 55965 to create a spectrum of IGR J17091\index{IGR J17091-3624} during the approximate time of its transition from a soft to a hard state \citep{Drave_Return}.  I fit this spectrum\index{Spectroscopy} above 2\,keV with a power-law, and extrapolated to find a 2--50\,keV flux of $8.56\times10^{-10}$\ergf .  Assuming that the transition bolometric correction factor $C_{Bol,Tran}$ is also $\lesssim3$, this corresponds to a bolometric flux of $\lesssim2.5\times10^{-9}$\ergf .
\par By comparing this with the results of \citealp{Maccarone_2pct} and \citealp{Altamirano_IGR_FH}, I find that IGR J17091\index{IGR J17091-3624} was likely emitting at no more than $\sim5$--20\% of its Eddington Limit\index{Eddington limit} at its peak.  This number becomes $\sim6\mbox{--}25$\% if I instead use $C_{Bol,Tran}=2.4$, or $\sim8\mbox{--}33$\% if $C_{Bol,Tran}=1.8$.  With this new range of values, I am able to re-derive the compact object\index{Compact object} mass as the function of the distance (Figure \ref{fig:IGRMass}).  I find that for a black hole\index{Black hole} mass of $\sim10$\ms , as suggested by \citealp{Iyer_Bayes}, IGR J17091 is within the Galaxy at a distance of 6--17\,kpc.  This is consistent with the estimated distance of $\sim11\mbox{--}17$\,kpc estimated by \citealp{Rodriguez_D} for a compact object mass of 10\ms .

\begin{figure}
    \includegraphics[width=\columnwidth, trim = 0mm 0mm 0mm 0mm]{images/MassDist.png}
    \captionsetup{singlelinecheck=off}
    \caption[Mass of the compact object in IGR J17091-3624 plotted against its distance, for values of peak Eddington fractions of $F_{Edd}=$0.05, 0.1, 0.2 and 0.33.]{Mass of the compact object\index{Compact object} in IGR J17091-3624\index{IGR J17091-3624} plotted against its distance, for values of peak Eddington\index{Eddington limit} fractions of $F_{Edd}=$0.05, 0.1, 0.2 and 0.33.}
   \label{fig:IGRMass}
\end{figure}

\subsection{Implications for Models of `Heartbeat' Variability}

\label{sec:IGRimp}

\par I have found that hysteretic\index{Hysteresis} HID\index{Hardness-intensity diagram} loops can execute in both directions in IGR J17091\index{IGR J17091-3624} (e.g. Section \ref{sec:Alta}), as well as found a revised estimate that IGR J17091 accretes\index{Accretion rate} at $\lesssim20$\% Eddington (Section \ref{sec:newmass}).  Both of these findings have implications for physical models of GRS 1915\index{GRS 1915+105}-like variability\index{Variability} in this source.
\par Firstly, I find that Eddington-limited\index{Eddington limit} accretion\index{Accretion} is neither necessary nor sufficient for GRS 1915-like\index{GRS 1915+105} variability\index{Variability}.  The discovery of GRS 1915-like variability in the sub-Eddington Rapid Burster\index{Rapid Burster} \citep{Bagnoli_RB,Bagnoli_PopStudy} provided the first evidence that Eddington-limited accretion may not be a driving factor in this type of variability.  I strengthen this case by finding that IGR J17091-3624 is also likely sub-Eddington.  As such, I further rule out any scenario in which Eddington-limited accretion is required for GRS 1915-like variability in black hole\index{Black hole} LMXBs\index{X-ray binary!Low mass} specifically.
\par Secondly, by using the direction of hysteretic\index{Hysteresis} HID\index{Hardness-intensity diagram} loops, I find that hard photon lag\index{Hard lag} in `heartbeat'\indexrho-like classes of IGR J17091\index{IGR J17091-3624} can be either positive or negative.  This could mean that we must rule out the causal connection between soft and hard emission being common to all classes.
\par In either case, I find that scenarios that require high global accretion rates\index{Accretion rate} or predict a consistent hard photon lag\index{Hard lag} (e.g. \citealp{Neilsen_GRSModel,Janiuk_Lag}), are not able to explain GRS 1915\index{GRS 1915+105}-like variability\index{Variability} in IGR J17091\index{IGR J17091-3624} unless they also feature geometric obscuration in a subset of variability classes.  I note that simulations by \citealp{Nayakshin_GRSModel} require an Eddington fraction of $\gtrsim0.26$ before GRS 1915-like variability occurs, a value which falls in the range $\sim0.05\mbox{--}0.33$ that I find for the peak Eddington fraction of IGR J17091.
\par An alternative way to explain the reversal of the direction of HID\index{Hardness-intensity diagram} hysteresis\index{Hysteresis} is by considering the information propagation timescales in GRS 1915\index{GRS 1915+105} and IGR J17091\index{IGR J17091-3624}.  A number of proposed models and scenarios to explain GRS 1915-like variability\index{Variability}, such as the scenario of \citet{Neilsen_GRSModel} which we describe in Section \ref{sec:Neilsen}, rely on information being propagated from one component of the LMXB\index{X-ray binary!Low mass} system to another; in the scenario of \citeauthor{Neilsen_GRSModel}, this propagation takes the form of a disk wind\index{Wind} which interacts with a geometrically displaced corona\index{Corona}.  Such a propagation takes a finite time.  If the timescale of the propagation of information is similar to or greater than the characteristic timescale of heartbeat\indexrho\ flares\index{Flare}, then each hard pulse from the corona could take place immediately before the flare subsequent to the flare which triggered it.  As heartbeats are a relatively coherent quasiperiodic\index{Quasi-periodic oscillation} phenomenon, it would appear to an observer that each hard pulse \textit{precedes} a soft flare, even though in reality the causality is reversed.  If this scenario was behind the hysteretic reversal seen in IGR J17091, then we would expect to see that only the fastest variability classes exhibited loops in the `wrong' direction, indicating soft lags.  However, I find that loops in Classes V\indexv, VI\indexvi, VII\indexvii\ and VIII\indexviii\ in IGR J17091 show hard lags\index{Hard lag}, whereas the slower classes III\indexiii\ and IV\indexiv\ show soft lags.  Therefore I rule out an information propagation timescale-based explanation for the difference in HID hysteresis between IGR J17091 and GRS 1915.
\par In addition to being near its Eddington limit\index{Eddington limit} GRS 1915\index{GRS 1915+105} also has the largest orbital period\index{Orbital period} of any known LMXB\index{X-ray binary!Low mass} (e.g. \citealp{McClintock_BHBs}).  \citealp{Sadowski_MagField} have also shown that thin, radiation dominated regions of disks\index{Accretion disk} in LMXBs require a large-scale threaded magnetic field\index{Magnetic field} to be stable, and the field strength required to stabilise such a disk in GRS 1915 is higher than for any other LMXB they studied.  I suggest that one of these parameters is more likely to be the criterion for GRS 1915-like variability\index{Variability}.  If better constraints can be placed on the disk size and minimum stabilising field strength in IGR J17091, it will become clear whether either of these parameters can be the unifying factor behind LMXBs that display GRS 1915-like variability\index{Variability}.

\section{Conclusions}

\par I have constructed the first model-independent set of variability classes\index{Variability class} for the entire portion of the 2011--2013 outburst\index{Outburst} of IGR J17091\index{IGR J17091-3624} that was observed with \indexrxte\rxte .  I find that the data are well-described by a set of 9 classes;  7 of these appear to have direct counterparts in GRS 1915\index{GRS 1915+105}, while two are, so far, unique to IGR J17091.  \textsf{D.A.} and I find that variability\index{Variability} in IGR J17091 is generally faster than in the corresponding classes of GRS 1915, and that patterns of quasi-periodic\index{Quasi-periodic oscillation} flares\index{Flare} and dips\index{Dip} form the basis of most variability in both objects.  Despite this, I find evidence that `heartbeat'\indexrho-like variability in both objects may be generated by different physical processes.  In particular, while hard photons always lag\index{Hard lag} soft in GRS 1915, I find evidence that hard photons can lag or precede soft photons in IGR J17091 depending on the variability class.
\par I also report on the long-term evolution of the 2011--2013 outburst\index{Outburst} of IGR J17091\index{IGR J17091-3624}, in particular noting the presence of 3 re-flares\index{Re-flare} during the later part of the outburst.  Using an empirical relation between hard\index{Low/Hard state}-soft\index{High/Soft state} transition luminosity and Eddington\index{Eddington limit} luminosity \citep{Maccarone_2pct}, I estimate that IGR J17091 was likely accreting\index{Accretion rate} at no greater than $\sim33$\% of its Eddington limit at peak luminosity.
\par I use these results to conclude that any model of GRS 1915\index{GRS 1915+105}-like variability\index{Variability} which requires a near-Eddington\index{Eddington limit} global accretion rate\index{Accretion rate} is insufficient to explain the variability we see in IGR J17091\index{IGR J17091-3624}.  As such I suggest that an extreme value of some different parameter, such as disk size or minimum stabilising large-scale magnetic field\index{Magnetic field}, may be the unifying factor behind all objects which display GRS 1915-like variability.  This would explain why sub-Eddington sources such as IGR J17091 and the Rapid Burster\index{Rapid Burster} do display GRS 1915-like variability, while other Eddington-limited sources such as GX 17+2\index{GX 17+2} and V404 Cyg\index{V404 Cyg} do not.