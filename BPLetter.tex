\chapter{The Bursting Pulsar GRO J1744-28: the Slowest Transitional Pulsar?}

\label{ch:BPletter}

\epigraph{\textit{I'll keep the sun behind us.  You've spent your entire life in the dark, I doubt that seeing something that bright would do you any good.}}{Lance Abell - \textit{Take the Sky}}
\vspace{1cm}

\par\noindent In Chapter \ref{ch:BPbig}, I performed a detailed analysis of all archival X-ray data of bursting\index{X-ray burst} behaviour in the Bursting Pulsar\index{Bursting Pulsar} (including \textit{RXTE}\indexrxte, \indexswift\textit{Swift}, \indexchandra\textit{Chandra}, \indexxmm\textit{XMM-Newton}, \indexsuzaku\textit{Suzaku}, \indexnustar\textit{NuStar}, and \indexintegral\textit{INTEGRAL}).  I found that the bursting phenomenology in the Bursting Pulsar is much richer than previously thought (e.g. \citealp{Giles_BP}): the characteristics of the bursts evolve with time and source luminosity. Near the end of this evolution, I observed periods of highly-structured and complex high-amplitude X-ray variability\index{Variability}.  I refer to this variability as `Structured Bursting'\index{Structured bursting}, and it is unlike what is seen in most other LMXBs\index{X-ray binary!Low mass}.
\par In Section \ref{sec:nuc}, I discuss the possibility that Structured Bursting\index{Structured bursting} is a manifestation of quasi-stable nuclear burning\index{Thermonuclear burning} on the surface of the neutron star\index{Neutron star}.  However as other types of burst\index{X-ray burst} can occur during periods of Structured Bursting without disrupting this behaviour (see e.g. Figure \ref{fig:meso_in_struc}), I consider this scenario to be unlikely.  As such, we must consider alternative explanations.  In this chapter I present the hypothesis that Structured Bursting is related to so-called `hiccup accretion'\index{Hiccup accretion}, a phenomenon seen in Transitional MilliSecond Pulsars\index{TMSP} (TMSPs).
\par \textbf{The results I present in this chapter have been published as \citet{BPletter}.}

\section{Transitional Millisecond Pulsars}

\par Millisecond Pulsars\index{Millisecond pulsar} are old radio pulsars\index{Pulsar} with spin\index{Spin} periods of order $\sim10$\,ms \citep{Backer_MSP}. They have long been believed to be the end product of systems containing a neutron star\index{Neutron star} in an LMXB\index{X-ray binary!Low mass}. In these systems, matter from a Roche-lobe\index{Roche lobe} overflowing star donates angular momentum\index{Angular momentum} to a Neutron star, spinning it up to frequencies of several 100 Hz \citep{Alpar_MSP}. A number of fast-spinning X-ray pulsars (accreting Millisecond Pulsars, or AMXPs\index{AMXP}\index{Accreting millisecond pulsar|see {AMXP}}) have been found in LMXBs (e.g. \citealp{Wijnands_XRPulsar,Altamirano_Broken,Patruno_AllAMXPs,Sanna_AMXP}), seemingly confirming this physical picture. At the end of this so-called `recycling'\index{Recycling} process, the system should transition from an accretion-powered pulsar to a rotation-powered pulsar. As such, it has long been expected that such a transition could be observed by finding a system which changes its character from an accreting Neutron star at one time to a radio pulsar at some later time. Subsequently a small family of 7 candidate objects have been discovered or proposed: these are referred to as Transitional Millisecond Pulsars (TMSPs)\index{TMSP}.
\par The first of these objects, \textbf{PSR J1023+0038}\index{PSR J1023+0038}, was identified by \citealp{Archibald_Link}. Although it appeared as a non-accreting radio pulsar\index{Pulsar} at the time of identification in 2009, previous optical studies showed that this system contained an accretion disk\index{Accretion disk} in 2002 \citep{Szkody_1023Accretion}. As such, the pulsar in this system must have switched from an accreting\index{Accretion} phase to a radio pulsar phase at some point between 2003 and 2009, confirming the identification of this system as a TMSP\index{TMSP}. The pulsar in this system has a spin\index{Spin} period of 1.69\,ms, and the companion\index{Companion star} is a star with a mass between $\sim$0.14--0.42\,M$_\odot$. \citealp{Archibald_Link} suggested that the low X-ray luminosity of PSR J1023+0038 in its accreting phase was due to accretion taking place in the propeller regime\index{Propeller effect} (see Section \ref{sec:prop}).  As previously discussed, whether a system is in the propeller regime depends on its spin\index{Spin} and its magnetic field strength\index{Magnetic field} (see also \citealp{Lewin_QPORev}). Additionally, below a certain accretion rate\index{Accretion rate}, no stable balance between ram pressure\index{Ram pressure} and radiation pressure\index{Radiation pressure} can form and any disk is ejected from the system (e.g. \citealp{Campana_NoDisk}). \citealp{Archibald_Link} suggested that the current accretion rate in PSR J1023+0038 is only slightly below this critical value, and that any small increase in accretion rate could cause accretion in this system to resume. They suggested the possibility of TMSP systems which flip back and forth between accreting and radio pulsar phases multiple times.
\par \citealp{Papitto_Swings} identified \textbf{IGR J18245-2452}\index{IGR J18245-2452} as the first known pulsar\index{Pulsar} to switch from a radio pulsar to an AMXP\index{AMXP} and back to a radio pulsar.  This source was first observed as a radio pulsar \citep{Manchester_PulsarCat}, before being observed several years later by \indexxmm\textit{XMM-Newton} \citep{Eckert_IGRJ18245} as an AMXP. Several months after the \textit{XMM-Newton} observation, \citealp{Papitto_Finding} found that the source had reactivated as a radio pulsar during X-ray quiescence\index{Quiescence}. The pulsar in this system has a period\index{Spin} of 3.93\,ms, and the companion star\index{Companion star} has a mass of $>0.17$\,M$_\odot$ \citep{Papitto_Swings}. During the 2013 outburst\index{Outburst} of IGR J18245-2452, \citealp{Ferrigno_TMSPVar} reported the presence of high-amplitude variability\index{Variability} in the X-ray lightcurve\index{Lightcurve}. They interpreted this as being due to the accretion rate\index{Accretion rate} $\dot{M}$ being very close to the critical rate at which the propeller effect\index{Propeller effect} begins to dominate the flow geometry. In this regime, small fluctuations in $\dot{M}$ cause so-called `hiccups'\index{Hiccup accretion}, in which matter alternates between being ejected by the propeller effect and being accreted onto the neutron star\index{Neutron star} poles (see our discussion of this effect in Section \ref{sec:hic}). Similar X-ray variability has subsequently been found in lightcurves from outbursts during the accreting phase of PSR J1023+0038\index{PSR J1023+0038} \citep{Bogdanov_TMSPVar}, suggesting that this variability is somehow intrinsic to TMSPs\index{TMSP} as a class of objects.
\par \textbf{1FGL J1227.9-4852}\index{1FGL J1227.9-4852} was first identified in the first \index{Fermi@\textit{Fermi}}\index{Fermi@\textit{Fermi}!LAT}\textit{Fermi}/LAT source catalogue \citep{Abdo_Catalogue}. \citealp{Hill_XSS} found that the $\gamma$-ray spectral\index{Spectroscopy} characteristics of this source are consistent with known millisecond radio pulsars\index{Pulsar}, although no radio pulsations were found. They suggested that this object could be associated with the X-ray source XSS J12270-4859\index{XSS J12270-4859|see {1FGL J1227.9-4852}}. Before 2009, XSS J12270-4859 showed optical emission lines typical of an accretion disk\index{Accretion disk} \citep{Pretorius_Optical}. \citealp{Hill_XSS} suggested that XSS J12270-4859 may also be a TMSP\index{TMSP}, which switched from an accreting \index{Accretion}phase to a radio pulsar millisecond pulsar phase between 2009 and 2011. Subsequent studies have found pulsations in both the radio \citep{Roy_12270Spin} and $\gamma$-ray \citep{Johnson_12270Spin} emissions of this source, confirming the system contains a pulsar and establishing its spin \index{Spin}period at 1.69\,ms.
\par \textbf{XMM J174457-2850.3}\index{XMM J174457-2850.3} is a neutron star\index{Neutron star} X-ray binary\index{X-ray binary}. Although no X-ray or radio pulsations\index{Pulsar} have been detected due to the faintness of the source, \citealp{Degenaar_174457} have found that the X-ray variability\index{Variability} properties of this source are similar to those seen in other TMSPs\index{TMSP}. This object also exhibits extended low-luminosity states during outbursts\index{Outburst}, which \citealp{Degenaar_174457} suggest may be symptomatic of TMSPs.
\par \textbf{3FGL J1544.6-1125}\index{3FGL J1544.6-1125}\index{1RXS J154439.4-112820|see {3FGL J1544.6-1125}} was also first identified in \index{Fermi@\textit{Fermi}}\index{Fermi@\textit{Fermi}!LAT}\textit{Fermi}/LAT data. \citealp{Bogdanov_Proxy} associated this object with the X-Ray source 1RXS J154439.4-112820. Due to the presence of $\gamma$-rays, as well as the presence of variability\index{Variability} in the X-ray lightcurve\index{Lightcurve} similar to IGR J18245-2452\index{IGR J18245-2452}, they proposed that this object is a TMSP\index{TMSP} in the accreting\index{Accretion} state. However, no pulsations\index{Pulsar} from this system have been detected in the X-ray or the radio, so the pulsar period is not known. \citealp{Bogdanov_Proxy} found a bimodality in count rate during the period of X-ray variability, suggesting that this behaviour can be explained as quick transitions between three quasi-stable accretion\index{Accretion} modes which they refer to as `low' , `high' and `flaring'. This effect has also been seen in the TMSP IGR J18245-2452\index{IGR J18245-2452} \citep{Ferrigno_TMSPVar}.
\par \citealp{Strader_6} identified the $\gamma$-ray source, \textbf{3FGL J0427.9-6704}\index{3FGL J0427.9-6704}, as a TMSP\index{TMSP}. They found that this source also displays X-ray variability\index{Variability} similar to what is seen from the other known TMSPs. Finally, \citealp{Rea_J0838} have proposed that the X-ray source \textbf{XMM J083850.4-282759}\index{XMM J083850.4-282759} may also be a TMSP. Although this source has not been detected in the gamma or the radio, the authors argued that X-ray variability coupled with X-ray flaring\index{Flare} seen from this object is reminiscent of similar behaviour seen in other TMSPs during subluminous disk\index{Accretion disk} states.
\par The phenomenology of currently known TMSPs\index{TMSP} is varied, and different methods have been used to conclude (or propose) that each individual system belongs to this class. The fact that 6 of the 7 objects show similar patterns of X-ray variability\index{Variability} during outburst\index{Outburst} suggests that this variability can be used as an indication that a system may be a TMSP.

\section{Comparison: TMSPs vs. the Bursting Pulsar}

\par \citealp{Rappaport_BPHistory} have previously suggested that the Bursting Pulsar\index{Bursting Pulsar} represents a slow X-ray pulsar\index{Pulsar} nearing the end of its accreting\index{Accretion} phase. As such it is natural to compare this system with TMSPs\index{TMSP}, which are also believed to be systems approaching this evolutionary stage. In addition to this, \citealp{Degenaar_174457} have previously noted that the Bursting Pulsar shows extended low-luminosity states during outburst\index{Outburst}, similar to those seen in the TMSP candidate XMM J174457-2850.3\index{XMM J174457-2850.3}.

\begin{figure*}
 \centering
 \resizebox{\columnwidth}{!}{\rotatebox{0}{\includegraphics[clip]{images/manylc_ann_diego.eps}}}
 \caption[Lightcurves from the Bursting Pulsar and from two TMSPs, showing similar patterns of variability.]{\textbf{Top:} 2--15\,keV \indexxmm\textit{XMM} lightcurve\index{Lightcurve} from the TMSP\index{TMSP} PSR J1023+0038\index{PSR J1023+0038}. \textbf{Middle:} 2--60\,keV \indexrxte\textit{RXTE} lightcurves from the Bursting Pulsar during its 1996 and 1997 outbursts\index{Outburst}, showing similar variability\index{Variability} patterns to those seen in PSR J1023+0038. \textbf{Bottom:} 2--15\,keV \textit{XMM} lightcurve from the TMSP IGR J18245-2452\index{IGR J18245-2452}. \textit{XMM} lightcurves are shown from 2--15\,keV so that they can be more directly compared with \textit{RXTE}.}
 \label{fig:lcs}
\end{figure*}

\par In Figure \ref{fig:lcs}, I show \indexrxte\textit{RXTE} lightcurves\index{Lightcurve} of `Structured Bursting'\index{Structured bursting} from the Bursting Pulsar\index{Bursting Pulsar} alongside lightcurves from periods of `hiccup'\index{Hiccup accretion} variability\index{Variability} observed in the confirmed TMSPs\index{TMSP} PSR J1023+0038\index{PSR J1023+0038} and IGR J18245-2452\index{IGR J18245-2452}. All three sources show similar patterns of X-ray variability\index{Variability}:
\begin{itemize}
\item \textit{Plateaus}\index{Plateau}: periods of approximately constant count rate with high-amplitude flicker noise (all plateaus in a given observation have approximately the same mean rate),
\item \textit{Dips}\index{Dip}: Periods of low count rate ($\lesssim0.5$ of the rate in plateaus) with significantly less flicker noise, and 
\item \textit{Flares}\index{Flare}: Relatively short-lived increases of the count rate to values $\gtrsim2$ times greater than the rate during plateaus.
\end{itemize}
In TMSPs\index{TMSP}, these features are interpreted as representing three quasi-stable accretion\index{Accretion} modes: the `high', `low' and `flaring' modes respectively (e.g. \citealp{Bogdanov_TMSPVar}). The most significant difference is that, in general, the variability in the Bursting Pulsar\index{Bursting Pulsar} occurs on timescales $\sim1$ order of magnitude longer than those in TMSPs.
\par In Figure \ref{fig:bimodal} I show histograms of the 1\,s-binned count-rate from all \indexrxte\textit{RXTE} observations of Structured Bursting\index{Structured bursting} in the 1996 (left) and 1997 (right) outbursts\index{Outburst} of the Bursting Pulsar\index{Bursting Pulsar}. As is the case for TMSPs\index{TMSP}, the histograms can be described with a number of log-Normally distributed populations: 3 populations in the 1996 outburst and 2 in the 1997 outburst. It is unclear why a population would be absent from the 1997 outburst, but some TMSPs have been observed to miss the `high' mode during hiccup accretion\index{Hiccup accretion} (e.g. IGR J18245-2452\index{IGR J18245-2452}, \citealp{Ferrigno_TMSPVar}).

\begin{figure}
 \centering
 \includegraphics[width=.82\linewidth, trim={1.3cm 0.1cm 1.7cm 1.1cm},clip]{images/hist_bo.eps}
 \caption[Histograms of the 1\,s binned count rates from all \textit{RXTE} observations of Structured Bursting in the 1996 and 1997 outbursts of the Bursting Pulsar.]{Histograms of the 1\,s binned count rates from all \indexrxte\textit{RXTE} observations of Structured Bursting\index{Structured bursting} in the 1996 (left) and 1997 (right) outbursts\index{Outburst} of the Bursting Pulsar\index{Bursting Pulsar}. For the 1996 outburst, I fit the distribution with three Gaussians, while for the 1997 outburst I fit the distribution with 2 Gaussians. The individual Gaussians are plotted in solid lines, while the combined total is plotted in a dashed line.}
 \label{fig:bimodal}
\end{figure}

\par Detailed works on the low and high modes observed in the lightcurves\index{Lightcurve} of TMSPs\index{TMSP} show that X-ray pulsations\index{Pulsar} are seen during both modes. Pulsations are fractionally weaker in the low state than the high state (for example varing between $4.0\pm0.2\%$ and $16.8\pm0.2\%$ in the TMSP IGR J18245-2452\index{IGR J18245-2452}, \citealp{Ferrigno_TMSPVar}). In the case of the Bursting Pulsar\index{Bursting Pulsar}, analysis by \textsf{A.S.} detects pulsations both during the low and the high modes; much like in TMSPs, the pulsations are weaker in the low mode. For example in \indexrxte\textit{RXTE} OBSID 10401-01-59-00 (in 1996), the pulsations had amplitudes of $3.5\pm0.2\%$ and $4.9\pm0.2\%$ respectively, while in OBSID 20078-01-23-00 (in 1997), the pulsations had amplitudes of $4.5\pm0.1\%$ and $6.0\pm0.1\%$ respectively. A reduction in pulse fraction in accreting pulsars\index{Pulsar} has been interpreted as a change in accretion\index{Accretion} geometry due to a sudden decrease in the amount of matter reaching the neutron star\index{Neutron star} (e.g. \citealp{Ibragimov_PulseFrac}), and as such this result provides direct evidence that the Structured Bursting\index{Structured bursting} in the Bursting Pulsar is caused by switches between accretion\index{Accretion} and propeller\index{Propeller effect}-driven outflows.

\par TMSPs\index{TMSP} are amongst the only LMXBs\index{X-ray binary!Low mass} which are also significant $\gamma$-ray sources (e.g. \citealp{Hill_XSS}). The \index{Fermi@\textit{Fermi}}\textit{Fermi} point source 3FGL J1746.3--2851c\index{3FGL J1746.3--2851c} is spatially coincident with the Bursting Pulsar\index{Bursting Pulsar}. While the field is too crowded to unambiguously associate 3FGL J1746.3--2851c with the Bursting Pulsar, the existence of a $\gamma$-ray point source at this location is consistent with the possibility that the Bursting Pulsar and TMSPs show the same phenomenology.

\par The spectral\index{Spectroscopy} evolution of known TMSPs\index{TMSP} is varied. In PSR J1023+0038\index{PSR J1023+0038}, the low, high and flaring modes all present similar spectra \citep{Bogdanov_TMSPVar}. However in IGR J18245-2452\index{IGR J18245-2452}, \citealp{Ferrigno_TMSPVar} have found a strong correlation between spectral hardness\index{Colour} and intensity during hiccups\index{Hiccup accretion}, showing that there is spectral evolution over time in this source. In Figure \ref{fig:HR} I show the hardness-intensity diagram\index{Hardness-intensity diagram} of the Bursting Pulsar\index{Bursting Pulsar} during periods of Structured Bursting\index{Structured bursting}. I find a significant correlation, similar to what is seen in IGR J18245-2452 \citep{Ferrigno_TMSPVar}. This is in contrast with other slow accreting pulsar systems such as Vela X-1\index{Vela X-1}, which show an anticorrelation between these parameters during periods of variability\index{Variability} \citep{Kreykenbohm_Vela}.

\begin{figure}
 \centering
 \includegraphics[width=.82\linewidth, trim={0.6cm 0.1cm 1.0cm 1.1cm},clip]{images/hr.eps}
 \caption[A 7--60/2--7\,keV hardness-intensity diagram for \textit{RXTE} observation 10401-01-59-00 of the Bursting Pulsar.]{A 7--60/2--7\,keV hardness-intensity\index{Hardness-intensity diagram} diagram for \indexrxte\textit{RXTE} observation 10401-01-59-00 of the Bursting Pulsar\index{Bursting Pulsar}; the lightcurve\index{Lightcurve} of this observation is shown in the inset. To correct for the high background\index{Background subtraction} of the region, I subtract the median count rate of \indexpca\textit{RXTE}/PCA observation 30075-01-24-00 from each band; at this time, the Bursting Pulsar was in quiescence\index{Quiescence}. I find a strong correlation between hardness\index{Colour} and count rate, with a Spearman Rank Correlation Coefficient\index{Spearman's rank correlation coefficient} of 0.93. Data for the hardness-intensity diagram are binned to 10\,s, while data for the lightcurve are binned to 5\,s.}
 \label{fig:HR}
\end{figure}

\section{Discussion}

\par In this chapter I compare the lightcurve\index{Lightcurve}, spectral\index{Spectroscopy} and timing properties of the Bursting Pulsar\index{Bursting Pulsar} at the end of its 1996 and 1997 outbursts\index{Outburst} with those observed from Transitional Millisecond Pulsars\index{TMSP}. The data suggest that the Bursting Pulsar may have undergone `hiccup'\index{Hiccup accretion} accretion similar to that seen in TMSPs, during which matter donated to the neutron star\index{Neutron star} by the companion star\index{Companion star} alternates between being accreted\index{Accretion} onto the poles of the neutron star\index{Neutron star} and being ejected from the system by the propeller\index{Propeller effect} effect (e.g. \citealp{Ferrigno_TMSPVar}). This similarity raises the exciting prospect of studying the physics of TMSPs in a completely different regime.
\par Recently \citealp{Campana_PropBorder} proposed a universal relation between magnetic moment\index{Magnetic moment}, spin frequency\index{Spin}, stellar radius and luminosity at the boundary between accretion\index{Accretion} and the propeller effect\index{Propeller effect}. Any object that exists on one side of this boundary should be able to accrete, whereas objects on the other side should be in the propeller phase or not accreting at all. In Figure \ref{fig:propBorder} I reproduce \citealp{Campana_PropBorder}'s results and include my estimates for the Bursting Pulsar\index{Bursting Pulsar} during the periods of Structured Bursting\index{Structured bursting}. I find that the Bursting Pulsar is consistent with lying on or near the boundary between propeller-mode and direct accretion, clustering with High Mass X-ray Binaries\index{X-ray binary!High mass} (as expected due to the Bursting Pulsar's high magnetic field\index{Magnetic field}), and supporting the link between `hiccups'\index{Hiccup accretion} and Structured Bursting.

\begin{figure}
 \centering
 \includegraphics[width=.82\linewidth, trim={0.6cm 0.1cm 1.0cm 1.1cm},clip]{images/propeff.eps}
 \caption[A plot of a number of objects ranging in scale from LMXBs and High-Mass X-ray Binaries (HMXBs) to Cataclysmic Variables (CVs) and Young Stellar Objects (YSOs). In each case, the object is plotted at the luminosity which defines its transition between propeller-mode accretion and free accretion.]{A plot of a number of objects ranging in scale from LMXBs\index{X-ray binary!Low mass} and High-Mass X-ray Binaries\index{X-ray binary!High mass} (HMXBs) to Cataclysmic Variables\index{Cataclysmic variable} (CVs) and Young Stellar Objects\index{Young stellar object}\index{YSO|see {Young stellar object}} (YSOs) (blue diamonds). In each case, the object is plotted at the luminosity which defines its transition between propeller-mode\index{Propeller effect} accretion\index{Accretion} and free accretion. \citealp{Campana_PropBorder} suggest that any object above the line of best fit accretes freely, whereas all objects below are in the propellor regime. The Bursting Pulsar\index{Bursting Pulsar} (red circle) is consistent with approaching this line during periods of Structured Bursting\index{Structured bursting}. Errorbars on the Bursting Pulsar represent the range of the reported magnetic fields\index{Magnetic field} as well as a range of stellar radii between 10--20\,km. The range in luminosity for the Bursting Pulsar is calculated using 1.5-25\,keV \indexpca\textit{RXTE}/PCA flux, assuming a distance of between 4--8\,kpc (e.g. \citealp{Kouveliotou_BP,Gosling_BPCompanion,Sanna_BP}) and a bolometric correction factor\index{Bolometric correction factor} of 1--3.  Data on the other objects taken from \citealp{Campana_PropBorder}. $L$ is the bolometric luminosity of the object in ergs\,s$^{-1}$, $P$ is the period in s, $R$ is the radius in cm and $\mu$ is the magnetic moment\index{Magnetic moment} in $Gauss\,cm^3$.}
 \label{fig:propBorder}
\end{figure}

\par If the `hiccups'\index{Hiccup accretion} in the Bursting Pulsar\index{Bursting Pulsar} show that the system is transiting to a radio pulsar, then the Bursting Pulsar should not lie in the $P$-$\dot{P}$ `graveyard'\index{Graveyard} region \citep[e.g.][]{vandenHeuvel_Graveyard}. To my knowledge, there is no measurment yet of the neutron star\index{Neutron star} spin down during the Bursting Pulsar\index{Bursting Pulsar}'s X-ray quiescent\index{Quiescence} state. Under the assumption that the Bursting Pulsar becomes a radio pulsar\index{Pulsar}, and that the possible spin down during that period is due to the same mechanism as those of the known radio pulsars, I can position the Bursting Pulsar in the $P$-$\dot{P}$ diagram (the plot of pulsar spin $P$ against spin-down rate $\dot{P}$, shown in Figure \ref{fig:graveyard}) by using the orbital period and estimates of its magnetic field\index{Magnetic field}. At $B\sim2\times10^{11}$G, the Bursting Pulsar falls well outside of the pulsar graveyard. I note that \citet{Pandey-Pommier_BPRad} and \citet{Russell_BPRad} did not detect a significant radio source at the location of the Bursting Pulsar during X-ray outburst\index{Outburst}. To my knowledge, there is no report of Radio detection/non-detection during X-ray quiescence.

\begin{figure}
 \centering
 \includegraphics[width=.82\linewidth, trim={0cm 0cm 0cm 0cm},clip]{images/ppdot.png}
 \caption[A plot of spin period against rate of spin-down for the population of all known radio pulsars, a so-called $P$--$\dot{P}$ diagram, showing where the Bursting Pulsar lies in this parameter space and indicating that it lies well outside of the pulsar `graveyard'.]{A plot of spin period against rate of spin-down for the population of all radio pulsars\index{Pulsar}, generally referred to as a $P$--$\dot{P}$ diagram.  Diagonal lines indicate the estimated age and surface magnetic field strength of a typical radio pulsar at any given position on the diagram.  Any pulsars to the right of the `graveyard' line on this plot are expected to be inactive in the radio, whereas objects to the left are expected to be observable as radio pulsars (e.g. \citealp{vandenHeuvel_Graveyard}).  The position of the Bursting Pulsar\index{Bursting Pulsar}, estimated from its surface field strength and spin period, is shown in red; well outside of the pulsar graveyard\index{Graveyard}.  Figure adapted from \citet{Lorimer_Handbook}, and is accurate as of the time of its first publication.}
 \label{fig:graveyard}
\end{figure}

\subsection{Comparison with other Objects}

\par In addition to the Bursting Pulsar\index{Bursting Pulsar}, several additional sub-10\,Hz accreting X-ray pulsars\index{Pulsar} have been discovered (e.g. GX 1+4\index{GX 1+4} and 4U 1626-67\index{4U 1626-67}, \citealp{Lewin_GX1,Rappaport_4U}). The reason behind the slow spins\index{Spin} of these objects is poorly understood, but a number of these systems have been seen to undergo `torque reversal'\index{Torque reversal} events, during which $\dot{P}$ switches sign (e.g. \citealp{Chakrabarty_4U,Chakrabarty_GX14}). In some sources, the magnitude of the spin-down during an event is of the same order of magnitude as the preceding period of spin-up, resulting in little or no net spin change. Torque reversal events occur irregularly, but the recurrence timescale\index{Recurrence time} varies between objects from weeks to decades (e.g. \citealp{Bildsten_Rev}).
\par The slow accreting pulsar\index{Pulsar} Vela X-1\index{Vela X-1} has been found to show an anticorrelation between hardness\index{Colour} and intensity \citep{Kreykenbohm_Vela}, whereas I find a strong positive correlation between these parameters in the Bursting Pulsar\index{Bursting Pulsar} during periods of Structured Bursting\index{Structured bursting} (Figure \ref{fig:HR}). This significant spectral\index{Spectroscopy} difference, combined with the other phenomenological differences between these objects reinforces the idea that the Bursting Pulsar exists in a very different physical state from the other known slow accreting pulsars.
\par Given that the Bursting Pulsar\index{Bursting Pulsar} has a strongly stripped stellar companion\index{Companion star} \citep{Bildsten_Nuclear}, a high magnetic field\index{Magnetic field} and shows significant spin-up during outburst\index{Outburst} (e.g. \citealp{Finger_BP,Sanna_BP}), it is difficult to explain its low spin\index{Spin} by suggesting the system is young or that the angular momentum\index{Angular momentum} transfer is inefficient. \citealp{Rappaport_BPHistory} suggest that the magnetic field and spin could be explained if much of the mass transfer in the system occurred before the primary became a neutron star\index{Neutron star}, but they note that this scenario is inconsistent with the low mass of the donor star.
\par Torque reversal\index{Torque reversal} events in the Bursting Pulsar\index{Bursting Pulsar} (similar to those seen in other slow accreting pulsars\index{Pulsar}, e.g. \citealp{Bildsten_Rev}) could explain why the pulsar has failed to reach a spin\index{Spin} rate on par with TMSPs\index{TMSP}.  Although no torque reversal event has been reported from the Bursting Pulsar, it is feasible that the recurrence timescale\index{Recurrence time} of such an event is longer than the $\sim20$ years for which the object has been studied (this is consistent with the recurrence timescales seen in other slow accreting pulsars). The discovery of torque reversal in the Bursting Pulsar would strongly link it with the other known slow accreting pulsars.

\par The Rapid Burster\index{Rapid Burster} is often compared to the Bursting Pulsar\index{Bursting Pulsar} due to the presence of regular Type II X-ray bursts\index{X-ray burst!Type II} in both objects (e.g. \citealp{Lewin_BP}). This system also contains an accreting neutron star\index{Neutron star}. \citealp{Iaria_RB} have suggested that the vast majority of matter transferred in this system is ejected, similar to a scenario suggested by \citealp{Degenaar_BPSpec} to explain high-velocity winds\index{Wind} from the Bursting Pulsar. However it remains unclear why the Rapid Buster does not show pulsations\index{Pulsar} or display the `hiccup'\index{Hiccup accretion} behaviour seen in the Bursting Pulsar.

\section{Conclusion}

\par The Bursting Pulsar\index{Bursting Pulsar} has a spin\index{Spin} rate $\sim2$ orders of magnitude less than previously known TMSPs\index{TMSP}, and a magnetic field\index{Magnetic field} $\sim2$ orders of magnitude stronger, but it still shows lightcurve\index{Lightcurve}, timing and spectral\index{Spectroscopy} behaviour which are remarkably similar to TMSPs. This raises the exciting prospect of exploring the physics of TMSPs in a previously unexplored physical regime. If the Bursting Pulsar itself is a transitional pulsar, it should emit radio pulsations during X-ray quiescence\index{Quiescence}. Future detections of radio pulsations from this object would unambiguously confirm it as a transitional pulsar.
