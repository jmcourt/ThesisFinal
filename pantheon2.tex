\chapter{PANTHEON suite}

\label{app:PAN}

\par In this section I present the headers and internal command lists of every piece of code in \textit{PANTHEON}\index{PANTHEON@\texttt{PANTHEON}} (Python ANalytical Tools for High-energy Event data manipulatiON), that suite of tools that I created and used during the work presented in this thesis.  The full code is available at \url{https://github.com/jmcourt/PANTHEON}.  PANTHEON makes use of the Astropy \citep{Astropy}, Matplotlib \citep{Hunter_MatPlotLib}, Numpy, Scipy \citep{NumPy} and Numba \citep{Numba}.

\section{FITS Genie}

\par \texttt{FITS Genie} is a script that allows the user to extract data from raw \texttt{FITS}\index{FITS@\texttt{FITS}} files.  The script was designed to interface with \indexrxte\textit{RXTE} data, but there is also limited implementation with \indexsuzaku\textit{Suzaku}.  The script produces \texttt{.plotd} and \texttt{.speca} files, which can be further processed with \textit{Plot Demon} and \textit{Spec Angel}.

\begin{minted}[fontsize=\scriptsize]{python}
#! /usr/bin/env python

# |----------------------------------------------------------------------|
# |-----------------------------FITS GENIE-------------------------------|
# |----------------------------------------------------------------------|

# Call as ./fitsgenie.py FILE1 PROD_REQ [LCHAN] [HCHAN] [BINNING] [FOURIER RES] 
#   [FOURIER SEP] [BGEST] [FLAVOUR]
#
# Takes 1 FITS Event file and produces .speca and .plotd formatted products to be
#   analysed by plotdemon and specangel.
#
# Arguments:
#
#  FILE1
#   The absolute path to the file to be used.
#
#  PROD_REQ
#   The products requested by the user.  The following inputs are valid:
#      'spec','speca','s' will cause FITSGenie to produce only a .speca file as output
#      'plot','plotd','p' will cause FITSGenie to produce only a .plotd file as output
#      'both','all','b','a','sp','ps' will cause both files to be output
#
#  [LCHAN]
#   Optional: The lowest channel on the PCA instrument on RXTE which will be used to
#     populate the data.  Default of 0 (minimum).
#
#  [HCHAN]
#   Optional: The highest channel on the PCA instrument on RXTE which will be used to
#     populate the data.  Default of 255 (maximum).
#
#  [BINNING]
#   Optional: The size, in seconds, of bins into which data will be sorted.  Takes the
#     value of the time resolution of the data if not specified by the user.  Default
#     of 2^-15s
#
#  [FOURIER RES]
#   Optional: The size of the individual time windows in which the data is to be split.
#   Fourier spectra will be made of each of these windows.  Default of 128s.
#
#  [FOURIER SEP]
#   Optional: The separation of the startpoints of individual time windows in which the
#     data is to be split.  Fourier spectra will be made of each of these windows.
#     Default of 128s.
#
#  [BGEST]
#   Optional: The approximate average background count rate during the observation in
#     cts/s.  Default of 30cts/s.
#
#  [FLAVOUR]
#   Optional: A useful bit of text to put on plots to help identify them later on.
\end{minted}

\section{Plot Demon}

\par \texttt{Plot Demon} is a script for manipulating the \texttt{.plotd} files output by \texttt{Fits Genie}, as well as time series stored in comma-separated variable files (CSVs).  The script allows the user to produce lightcurves\index{Lightcurve}, hardness-intensity diagrams\index{Hardness-intensity diagram} and colour-colour diagrams\index{Colour-colour diagram}, among other products.  The script also allows the user to modify the data by rebinning, clipping or folding\index{Folding} it using the algorithms I detail in Section \ref{sec:vfold}.

\begin{minted}[fontsize=\scriptsize]{python}
#! /usr/bin/env python

# |----------------------------------------------------------------------|
# |------------------------------PLOT DEMON------------------------------|
# |----------------------------------------------------------------------|

# Call as ./plotdemon.py FILE1 [FILE2] [FILE3] BINNING
#
# Takes 1-3 .plotd files and plots relevant astrometric plots
#
# Arguments:
#
#  FILE1
#   The absolute path to the first file to be used (generally the lowest energy band)
#
#  [FILE2]
#   The absolute path to the second file to be used
#
#  [FILE3]
#   The absolute path to the third file to be used (generally the highest energy band)
#
#  [BINNING]
#   Optional: the size, in seconds, of bins into which data will be sorted.

def give_inst():                                      # Define printing this list of
                                                      # instructions as a function
   print 'COMMANDS: Enter a command to manipulate data.'
   print ''
   print 'DATA:'
   print '* "rebin" to reset the data and load it with a different binning.'
   print '* "clip" to clip the data.'
   print '* "norm time" to renormalise the times by the start time of the data'
   print '* "mask" to remove a range of data.'
   print '* "rms" to return the fractional rms of the data.'
   print '* "fold" to fold data over a period of your choosing'+(' (requires PyAstron'+
         'omy module!)' if not module_pyastro else '')+'.'
   print '* "autofold" to automatically seek a period over which to fold data'+(' (re'+
         'quires PyAstronomy module!)' if not module_pyastro else '')+'.'
   print '* "varifold" to fold over a non-constant period using an algorithm optimise'+
         'd for high-amplitude quasi-periodic flares.'
   print '* "plot bursts" to plot the results of the peak-finding algorithm used in v'+
         'arifold.'
   print ''
   print '1+ DATASET PLOTS:'
   print '* "lc" to plot a simple graph of flux over time.'
   print '* "bg" to plot background over time, if background has been estimated for t'+
         'hese files.'
   print '* "animate" to create an animation of the lightcurve as the binning is incr'+
         'eased.'
   print '* "circanim" to create an animation of the lightcurve circularly folded as '+
         'the period is increased.'
   print '* "lombscargle" to create a Lomb-Scargle periodogram of the lightcurve.'
   print '* "autocor" to plot the auto-correlation function.'
   print '* "rmsflux" to plot the rms-flux relationship of the data.'
   if nfiles>1:                                       # Only display 2-data-set inst-
                                                      # -ructions if 2+ datasets given
      print ''
      print '2+ DATASET PLOTS:'
      print '* "hardness21" to plot a hardness/time diagram of file2/file1 colour ove'+
            'r time.'
      print '* "hardness12" to plot a hardness/time diagram of file1/file2 colour ove'+
            'r time.'
      print '* "hid21" to plot a hardness-intensity diagram of file2/file1 colour aga'+
            'inst total flux.'
      print '* "hid12" to plot a hardness-intensity diagram of file1/file2 colour aga'+
            'inst total flux.'
      print '* "calcloop21" to return the probability of a null hysteresis in the 12 '+
            'HID.'
      print '* "col21" to plot file2/file1 colour against time.'
      print '* "col12" to plot file1/file2 colour against time.'
      print '* "band" to plot the lightcurve of a single energy band.'
      print '* "bands" to plot lightcurves of all bands on adjacent axes.'
      print '* "xbands" to plot lightcurves of all bands on the same axes.'
      print '* "compbands21" to plot lightcurves of bands 2 and 1 against each other.'
      print '* "crosscor21" to plot the cross-correlation function of band 1 with ban'+
            'd 2.'
      print '* "timeres crosscor21" to plot the time-resolved cross-correlation funct'+
            'ion of band 1 with band 2' 
      print '* "all" to plot all available data products.'
   if nfiles==3:                                      # Only display 3-data-set inst-
                                                      # -ructions if 3 datasets given
      print ''
      print '3 DATASET PLOTS:'
      print '* "hardness32" to plot a hardness/time diagram of file3/file2 colour ove'+
            'r time.'
      print '* "hardness23" to plot a hardness/time diagram of file2/file3 colour ove'+
            'r time.'
      print '* "hardness31" to plot a hardness/time diagram of file3/file1 colour ove'+
            'r time.'
      print '* "hardness13" to plot a hardness/time diagram of file1/file3 colour ove'+
            'r time.'
      print '* "hid32" to plot a hardness-intensity diagram of file3/file2 colour aga'+
            'inst total flux.'
      print '* "hid23" to plot a hardness-intensity diagram of file2/file3 colour aga'+
            'inst total flux.'
      print '* "calcloop32" to return the probability of a null hysteresis in the 32 '+
            'HID.'
      print '* "hid31" to plot a hardness-intensity diagram of file3/file1 colour aga'+
            'inst total flux.'
      print '* "hid13" to plot a hardness-intensity diagram of file1/file3 colour aga'+
            'inst total flux.'
      print '* "calcloop31" to return the probability of a null hysteresis in the 31 '+
            'HID.'
      print '* "col32" to plot file3/file2 colour against time.'
      print '* "col23" to plot file2/file3 colour against time.'
      print '* "col31" to plot file3/file1 colour against time.'
      print '* "col13" to plot file1/file3 colour against time.'
      print '* "compbands31" to plot lightcurves of bands 3 and 1 against each other.'
      print '* "compbands32" to plot lightcurves of bands 3 and 2 against each other.'
      print '* "ccd" to plot a colour-colour diagram (3/1 colour against 2/1 colour).'
      print '* "timeres crosscor31" to plot the time-resolved cross-correlation funct'+
            'ion of band 3 with band 1'
      print '* "timeres crosscor32" to plot the time-resolved cross-correlation funct'+
            'ion of band 3 with band 2'
      print '* "crosscor31" to plot the cross-correlation function of band 3 with ban'+
            'd 1.'
      print '* "crosscor32" to plot the cross-correlation function of band 3 with ban'+
            'd 2.'
   print ''
   print 'BURST ANALYSIS:'
   print '* "burst get" to interactively extract burst data for analysis.'
   print '* "burst peaks" for a histogram of peak heights of extracted bursts.'
   print '* "burst risetimes" for a histogram of rise times of extracted bursts.'
   print '* "burst falltimes" for a histogram of fall times of extracted bursts.'
   print '* "burst lengths" for a histogram of durations of extracted bursts.'
   print '* "burst help" for further information on burst analysis.'
   print ''
   print 'SAVING DATA TO ASCII:'
   print '* "export" to dump the lightcurve and colour data into an ASCII file.'
   print '* "bgdump" to export background lightcurve to an ASCII file.'
   print '* "timenorm" to toggle absolute or relative time values on x-axis.'
   print ''
   print 'TOGGLE OPTIONS:'
   print '* "errors" to toggle whether to display errors in plots.'
   print '* "lines" to toggle lines joining points in graphs.'
   print '* "ckey" to toggle colour key (red-blue) for the first five points in all p'+
         'lots.'
   print '* "save" to save to disk any plots which would otherwise be shown.'
   print ''
   print 'ADVANCED OPTIONS:'
   print '* "burstalg" to select algorithm for finding pulse peaks in lightcurve.'
   print ''
   print 'OTHER COMMANDS:'
   print '* "info" to display a list of facts and figures about the current PlotDemon'+
         ' session.'
   print '* "reflav" to rewrite the flavour text used for graph titles.'
   print '* "help" or "?" to display this list of instructions again.'
   print '* "quit" to quit.'

give_inst()                                          # Print the list of instructions
\end{minted}

\section{Spec Angel}

\par \texttt{Spec Angel} is a script to allow users to produce power spectra\index{Fourier analysis} from \texttt{.speca} files output by \texttt{Fits Genie}.  These power spectra can be linearly or logarithmically binned, and can be normalised in a number of different ways.

\begin{minted}[fontsize=\scriptsize]{python}

#! /usr/bin/env python

# |----------------------------------------------------------------------|
# |------------------------------SPEC ANGEL------------------------------|
# |----------------------------------------------------------------------|

# Call as ./specangel.py FILE1 [LBINNING]

# Takes 1 RXTE FITS Event file and produces an interactive spectrogram
#
# Arguments:
#
#  FILE1
#   The absolute path to the file to be used.
#
#  [LBINNING]
#   Optional- the logarithmic binning factor 'x'; frequency data will be binned into 
#   bins which have their lefthand edges defined by the formula 10**(ix) for integer i.
#

def give_inst():                                      # Define printing this list of
                                                      # instructions as a function
   print 'COMMANDS: Enter a command to manipulate data.'
   print ''
   print 'DATA:'
   print '* "rebin" to reset the data and load it with a different normalisation and '+
         ' binning.'
   print '* "clip" to clip the range of data.'
   print '* "reset" to reset data.'
   print ''
   print 'SPECTROGRAM:'
   print '* "sg plot" to plot the spectrogram currently being worked on.'
   print '* "sg floor" to set a minimum value for the spectrogram'+"'"+'s z-axis colo'+
         'ur key.'
   print '* "sg ceil" to set a maximum value for the spectrogram'+"'"+'s z-axis colou'+
         'r key.'
   print '* "sg auto" to automatically set colour floor and ceiling.'
   print '* "sg log" to toggle logarithmic spectrogram plotting.'
   print ''
   print 'POWER SPECTRA:'
   print '* "aspec" to plot the average spectrum and return the frequency of its high'+
         'est peak.'
   print '* "gspec" to get an individual spectrum at any time and plot it.'
   print '* "peaks" to plot a graph of the frequency of the strongest oscillation aga'+
         'inst time.'
   print '* "rates" to get a simple lightcurve of the data.'
   print '* "fqflux" to plot "peaks" against "rates".'
   print ''
   print 'TOGGLE OPTIONS:'
   print '* "errors" to toggle errorbars on power spectra plots.'
   print '* "save" to save to disk any plots which would otherwise be shown.'
   print ''
   print 'OTHER COMMANDS:'
   print '* "info" to display a list of facts and figures about the current SpecAngel'+
         ' session.'
   print '* "reflav" to rewrite the flavour text used for graph titles.'
   print '* "export" to create an ASCII file of the average power density spectrum.'
   print '* "help" or "?" to display this list of instructions again.'
   print '* "quit" to Quit'

give_inst()                                          # Print the list of instructions
\end{minted}

\section{Back Hydra}

\par \texttt{Back Hydra} is a script to subtract the background\index{Background subtraction} from the data in a \texttt{.plotd} file.  It requires a background lightcurve created by the \texttt{pcabackest} tool available from NASA's \texttt{FTOOLS}\index{FTOOLS@\texttt{FTOOLS}} suite.

\begin{minted}[fontsize=\scriptsize]{python}
#! /usr/bin/env python

# |----------------------------------------------------------------------|
# |-----------------------------BACK HYDRA-------------------------------|
# |----------------------------------------------------------------------|

# Call as ./bckghydra.py DATA_FILE BACK_FILE SAVE_FILE

# Takes a .plotd file and a background file created with PCABACKEST and returns
#
# Arguments:
#
#  DATA_FILE
#   The absolute path to the file to be used as data.
#
#  BACK_FILE
#   The file to be used as background; does not need to be the same binning as File 1.
#   suggest using pcabackest from FTOOLS to produce this file.
#   FTOOLS can be found at http://heasarc.gsfc.nasa.gov/ftools/
#
#  SAVE_FILE
#   The location to save the resultant background-subtracted file
#
\end{minted}

\section{PAN Lib}

\par \texttt{PAN Lib} is a library of functions.  I use this library extensively in the other scripts in this suite.

\begin{minted}[fontsize=\scriptsize]{python}
#! /usr/bin/env python

# |----------------------------------------------------------------------|
# |-------------------------------PAN_LIB--------------------------------|
# |----------------------------------------------------------------------|

# A selection of useful functions which are placed here to reduce clutter in the other
#  files of

# PANTHEON.
#
# Contents:
#
# ARGCHECK   - compares the list of arguments against a value given as the minimum
#              allowed number of arguments.  If the list of arguments is too short,
#              throw a warning and kill the script.
#
#  BINIFY    - takes a x-series with its associated y-axis data and y-axis errors.
#              Rebins the data into larger linear bins with a width of the user's
#              choosing, and returns the tuple x,y,y_error.
#
#  BOOLVAL   - takes a list of Boolean values and, interpreting it as binary, returns 
#              its integer value.
#
#  EQRANGE   -
#
#  EVAL_BURST-
#
#  FILENAMECHECK  - checks to see whether a proposed input file has the correct file
#                   extension.
#
#  FOLDIFY   - takes a time series with its associated y-axis data and y-axis errors. 
#              Folds this data over a time period of the user's choosing, and returns
#              them as the tuple x,y,y_error.
#
#  FOLD_BURSTS - uses GET_BURSTS to obtain burst locations then interpolates to
#                populate phase information for all other points
#
#  GET_BURSTS- takes an array of data, looks for bursts and returns an array of tuples
#              containing the start and end points of these bursts.
#
#  GET_DIP   - returns the index of the lowest point between two user-defined flags in
#              a dataset.
#
#  GTIMASK   - returns a data mask when given a time series and a GTI object
#
#  LBINIFY   - takes a linearly binned x-series with associated y-axis data and y-axis
#              errors and rebins them into bins of a constant width in logx space.
#              In places where the logarithmic bins would be finer than the linear
#              bins, the linear bins are retained.
#
#  LEAHYN    - takes the raw power spectrum output from the scipy FFT algorithm and
#              normalises it using Leahy normalisation.
#
#  LH2RMS    - takes a Leahy-normalised power spectrum and converts it to an
#              (RMS/Mean)^2-normalised power spectrum.
#
#  LHCONST   - returns the normalisation of the white noise component in a Leahy-
#              normalised power spectrum with no features in the range 1.5kHz - 4kHz.
# 
#
#  MXREBIN   - takes a 2-dimensional set of data and corresponding errors linearly
#              binned on the x-axis and rebins them by an integer binning factor of
#              the user's choice.
#
#  NONES     - like np.zeros, but with None.
#
#  PDCOLEX   - extracts colours from a set of 2 or 3 lightcurves
#
#  PLOTDLD   - load and unpickle a .plotd file and extract its data.
#
#  PLOTDSV   - collect a selection of data products as a library, pickle it and save
#              as a .plotd file.
#
#  RMS_N     - takes the raw power spectrum output from the scipy FFT algorithm and
#              normalises it using (RMS/Mean)^2 normalisation.
#
#  SAFE_DIV  - Divides two arrays by each other, replacing NaNs that would be caused
#              by div 0 errors with zeroes.
#
#  SIGNOFF   - prints an dividing line with some space.  That's all it does.
#
#  SINFROMCOS- calculates the sines of an array of values when also passed their co-
#              sines.  If both sines and cosines of the array are required, this method
#              is faster than calling both trig functions.
#              Also contains function COSFROMSIN.
#
#  SLPLOT    - plots an x-y line plot of two sets of data, and then below plots the
#              same data on another set of axes in log-log space.
#
#  SPECALD   - load and unpickle a .speca file and extract its data.
# 
#  SPECASV   - collect a selection of data products as a library, pickle it and save
#              as a .speca file.
#
#  SRINR     - calculates whether a value given by a user is within an existant evenly
#              spaced array and, if it is, returns the index value of the closest
#              match to this value within the array.
#              Intended for validating subranges specified by user.
#
#  TNORM     - takes a list of times, and subtracts the lowest value from each entry
#              such that a new list starting with 0 is produced.  Large number
#              subtraction errors are avoided by checking that every entry is an
#              integer number of time-resolution steps from zero.
#
#  UNIQFNAME - checks if a proposed filename is currently in use and, if so, proposes
#              an alternative filename to prevent overwrite.
#
#  XTRFILLOC - takes a filepath and outputs the file name and its absolute(ish)
#              location
#
\end{minted}