\chapter{Conclusions}

\epigraph{\textit{And this goes on and on and back and forth for 90 or so minutes, until it just sort of... ends.}}{Dennis Reynolds -- \textit{It's Always Sunny in Philadelphia}}

\vspace{1cm}


\par\noindent In this thesis, I have presented the results of phenomenological studies of the X-ray variability\index{Variability} seen in two unusual LMXBs\index{X-ray binary!Low mass}: IGR J17091-3624\index{IGR J17091-3624} and GRO J1744-28\index{Bursting Pulsar} (the ``Bursting Pulsar'').  I have analysed these results in the context of previous studies of variability in GRS 1915+105\index{GRS 1915+105} and MXB 1730-335\index{Rapid Burster} (the ``Rapid Burster''), systems which are often compared to IGR J17091 and the Bursting Pulsar respectively.  In doing so I have discovered a number of new similarities and differences between these objects.  On the back of this analysis, I have evaluated the physical models and scenarios which have been proposed to explain the variability in these objects.  In doing so, I have brought us closer to an understanding of the accretion\index{Accretion} physics that underlies this exotic behaviour.
\par In Chapter \ref{ch:IGR}, I have presented a new set of variability classes\index{Variability class} to describe IGR J17091\index{IGR J17091-3624}: these classes are analogous to, but independent from, the classes presented by \citet{Belloni_GRS_MI} to describe GRS 1915\index{GRS 1915+105}.  Comparing my set of variability classes to those of \citeauthor{Belloni_GRS_MI}, I found a number of variability classes which are only seen in one of the two objects, as well as a number of types of variability which are seen in both.  When studying the spectral\index{Spectroscopy} timing properties of both objects, I found another significant difference: while hard photons lag\index{Hard lag} soft photons in every variability class of GRS 1915, the sign of this lag varies from class to class in IGR J17091.  This finding rules out any physical picture in which the hard lag is caused by a corona\index{Corona} reacting to changes in the flux from the disk\index{Accretion disk}, instead suggesting that the hard lag is generated by a spectral change in the emission from the disk.
\par In Chapter \ref{ch:BPbig}, I have presented a new set of classifications for Type II-like\index{X-ray burst!Type II} X-ray bursts\index{X-ray burst} from the Bursting Pulsar\index{Bursting Pulsar}.  In doing so, I have discovered previously unreported bursting behaviour in the late stages of outbursts\index{Outburst} of the Bursting Pulsar, namely `Mesobursts'\index{Mesoburst} and `Structured Bursting'\index{Structured bursting}.  I find that Mesobursts may be a manifestation of quasi-stable thermonuclear burning\index{Thermonuclear burning} on the surface of the neutron star\index{Neutron star}; a phenomenon which has long been predicted to occur on the Bursting Pulsar but which has never been conclusively identified (e.g. \citealp{Bildsten_Nuclear}).  There are similarities between lightcurves\index{Lightcurve} of Structured Bursting and the lightcurves of another form of quasi-stable nuclear burning predicted by \citet{Heger_MargStab}.  However I find that at least one Mesoburst occured during a period of Structured Bursting without disrupting it, suggesting that Structured Bursting is non-nuclear in nature.  Instead, in Chapter \ref{ch:BPletter} I have identified similarities between Structured Bursting and variability\index{Variability} seen in Transitional Millisecond Pulsars\index{TMSP}.  This raises the possibility that Structured Bursting is a manifestation of `hiccup' accretion\index{Hiccup accretion}; spasmodic accretion onto a neutron star caused by small perturbations of its magnetospheric radius\index{Magnetospheric radius} near the boundary of the propeller regime\index{Propeller effect}.
\par From a phenomenological standpoint, I find that the variability\index{Variability} seen in IGR J17091\index{IGR J17091-3624} and the Bursting Pulsar\index{Bursting Pulsar} is generally even more complex than previously thought.  IGR J17091 and the Bursting Pulsar have often been considered `twin systems' of GRS 1915\index{GRS 1915+105} and the Rapid Burster\index{Rapid Burster} respectively, but I find a number of differences between each pair of twins which makes such a simple picture seem unlikely.  Variability in GRS 1915 has traditionally been thought to be tied to its near-Eddington\index{Eddington limit} accretion rate\index{Accretion rate}; however, I find that IGR J17091 likely accretes at $\lesssim33$\% of its Eddington Limit.  The Rapid Burster shows Type II\index{X-ray burst!Type II} bursts which transition smoothly between 2 classes over the course of an outburst\index{Outburst}, whereas I find that bursts in the Bursting Pulsar can be described in no less than 4 classes which take place at different periods of each outburst.  Rather than suggesting that these pairs of objects are unrelated, I suggest that further study of their differences will lead to better understanding of the physics behind the instabilities\index{Instability} that they present.
\par In Chapter \ref{ch:conc} I discuss the relationship between GRS 1915-like\index{GRS 1915+105}\index{Variability} variability and Type II X-ray bursts\index{X-ray burst!Type II}.  While there are a number of problems with assuming that these two types of variability are the same, I find a number of similarities between them that suggest at least some of the physics underlying these phenomena are similar.  Finally, I suggest that phase-resolved spectral\index{Spectroscopy!Phase-resolved} studies by the next generation of space telescopes will allow us to fully understand the relationships, or lack thereof, between these four enigmatic objects.
\par I also present a number of results unconnected to the variability\index{Variability} seen in these objects.  In Chapter \ref{ch:IGR} I provide new constraints on the distance of IGR J17091-3624\index{IGR J17091-3624} and the mass of its black hole\index{Black hole}, and I also present the first \textit{INTEGRAL}\indexintegral\ detection of the object above 150\,keV.  Additionally in Chapters \ref{ch:IGR} and \ref{ch:BPbig} I report the discovery of `re-flares'\index{Re-flare} in the tails of outbursts\index{Outburst} in both IGR J17091 and the Bursting Pulsar\index{Bursting Pulsar}.  I have also created a number of algorithms to identify bursts\index{X-ray burst} or flares\index{Flare}, and to `fold'\index{Folding} datasets which show repeating variability\index{Variability} with a non-constant frequency (Chapter \ref{ch:methods}).  These are encoded as part of my own suite of computational tools to analyse X-ray data (\texttt{PANTHEON}\index{PANTHEON@\texttt{PANTHEON}}, see Appendix \ref{app:PAN}).
\par In conclusion, the work I present in this thesis provides a comprehensive framework for future study of variability\index{Variability} in IGR J17091\index{IGR J17091-3624} and the Bursting Pulsar\index{Bursting Pulsar}.  Using this framework, I have been able to rule out a number of models and physical scenarios which have been proposed to explain the behaviour seen in these systems.  Further studies of the key properties of these systems will allow us to better understand the exotic instabilities\index{Instability} which can be present in accretion disks\index{Accretion disk} and, as such, improve our knowledge of the physics of accretion\index{Accretion} in general.